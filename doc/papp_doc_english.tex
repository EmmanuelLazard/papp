%
% PAPP Documentation , in LaTeX.
%
% $Id: papp_doc.tex,v 1.19 1995/04/09 18:17:39 bousch Exp $
%
%\documentstyle[11pt]{article}
\documentclass[10pt]{article} 
%\usepackage{a4wide}
%\usepackage{times}

%\french
\def\contentsname{Table of contents}
\parskip 1ex plus 0.5ex minus 0.5ex
\tolerance 1000

% Macro pour inserer du code TeX non utilise pour la version HTML
\def\tthdump#1{#1}

% Macro pour ecrire du code different en HTML (branche iftth)
% et en TeX (branche \else). Usage :
% \iftth  code HTML   \else   code Tex \fi
\newif\iftth

% \times semble manquer dans TtH, on le definit
\def\fois
{\iftth 
   {\begin{html}<font face=symbol> Ž </font>\end{html}} 
\else 
   \times 
\fi}  

   

\begin{document}

\title{\huge\bf Documentation for Papp 1.22}
\author{ {Thierry Bousch {\tt <bousch@topo.math.u-psud.fr>}}
\and {Liste Papp-FFO {\tt <ffo-papp@egroups.fr>}}
}
\date{September 2000}
\maketitle


\iftth   
{\begin{html}<small>If mathematics display oddly in that document,
        change the encoding (MacRoman or ISO-8859-1) in 
        your browser. If this is not enough, you might 
        find a solution there :
        http://hutchinson.belmont.ma.us/tth/manual/</small>
 \end{html}} 
\else {}
\fi    


\section{Introduction}

	PAPP is a pairing program, that is, something which is supposed to 
simplify referees work during tournaments.  With it, they can 
inscribe players, save the results of the rounds, calculate pairings 
automatically and produce files used to calculate a international 
rating list (\`a la Elo).

	PAPP also offers a mini-database to allow players numbers 
management, and the pairing procedure is completely parametrable.

	PAPP is written in ANSI C, especially for the GCC. The manual has 
been written in \LaTeX. The lexical and syntactic analysis phases use 
Flex and Bison, from the ''Free Software Foundation''.

\section{First contact}

	The program is named \verb|papp| or \verb|papp.exe|, depending on the
exploitation system. Launch it; after a couple of seconds, the main 
menu appears.

	All commands are accessible with a single key stroke.  It is not 
necessary to type [Return] afterwards, nor is it necessary after the 
other mennus, nor after the Yes/No questions.

	Experiment a little bit, it is not dangerous.  PAPP will emit a 
protestation beep if you press an invalid key (raise the sound level 
of your computer if necessary). Most commands will be uneffective 
since no player is inscribed in the tournament, so type `\verb|I|' to 
inscribe players.

\subsection{Inscriptions}

Faced with the prompt \verb|Nom & Prenom:|, type `\verb|y|', then hit 
the [Tab] key; PAPP then lists all players whose name begins with 
`y', That is \verb|YASOJIMA Keiji| 
and \verb|YASUI Tsugo|.  Note that the input buffer now contains 
\verb|YAS|, because PAPP has completed the name until the first 
ambiguity --- here, it does not know wether the fourth letter is `O' 
or `U'.  Delete the buffer with [Backspace] ou [Esc], and type 
\verb|laz|, then [Tab].  This time, there is only one possible 
completion: \verb|LAZARD Emmanuel|, and you only have to confirm with 
[Return].

	Now, enter the name of a new, unknown player (use [Tab]
to verify that he/she is not already in the database).  PAPP will
ask for confirmation (you may have done a typographic error), then 
the player's nationnality (just hit [Return] if it is the default 
country) so that he can choose a number for the new player  --- you 
can change that number if you wish.  The player will then be added to 
the database.

Spend some time playing with the database and inscribing players.  
Once you get bored, just type [Return] (or Ctrl-C) at the prompt.  
Then, at the main menu, you can check the list of players in the 
tournament with `\verb|L|'.

\subsection{Pairings}

If you hit `\verb|V|', you'll see the list of all these players, one 
per line.  That's because they are not yet paired.  It is possible to 
pair them manually with `\verb|M|', or to ask PAPP to pair 
automatically all the players that are not yet paired with `\verb|A|'.  
So type `\verb|A|': the list of pairings appears shortly alfterwards; 
This time, there are two players on each line (except on the last line 
if there is an odd number of players).  This means the pairings are 
complete.

If some players show up late after the beginning of round one, the 
tournament director has the choice to accept them or not; but as far 
as PAPP is concerned, there is no difficulty: you come back to the 
main menu, you inscribe the players, and you type `\verb|A|' to 
complete the pairings.  It is possible to inscribe players for a 
particular round as long as all the results for that round have not 
been entered and validated, which should be flexible enough for most 
cases.

\subsection{Entering results}

When the games of the round end and you begin to collect score coupons 
on your desk, select the option `\verb|R|' to enter the results.  the 
prompt \verb|Joueur & Score:| is displayed in an empty screen; to 
enter a game result, type in the name (or the beginning of the name , 
or the number) of one of the players, followed by his/her score, 
separated by a space.  PAPP verifies the name or the number of the 
designated player, outputs a warning message if the result had already 
been entered, then displays the filled coupon.  En cas d'erreur, vous 
pouvez {\em effacer} un coupon en tapant le signe `\verb|-|' suivi du 
nom de l'un des joueurs de la partie (ou, si vous pr\'ef\'erez, en 
tapant le num\'ero de l'un des joueurs suivi de la lettre `\verb|e|' 
--- l'espace entre les deux est facultatif).  Il est possible d'entrer 
les scores sous forme relative; ainsi, `\verb|145+4|', `\verb|145=|' 
et `\verb|145-6|' signifient que le joueur 145 a fait quatre pions de 
plus que son adversaire, a \'egalis\'e, ou a perdu de six pions, 
respectivement --- ici encore, il n'est pas n\'ecessaire de s\'eparer 
le num\'ero du joueur de son score, parce qu'il n'y a pas 
d'ambigu{\"\i}t\'e possible.

Si vous n'\^etes pas s\^ur des informations port\'ees sur un coupon 
(notamment l'orthographe ou le num\'ero Elo d'un joueur), ou si vous 
n'avez pas de coupon du tout, par exemple parce qu'on vous a donn\'e 
le r\'esultat oralement, tapez sur [Tab]: PAPP affichera alors tous 
les coupons, d'abord ceux qui ont \'et\'e entr\'es, puis ceux qui 
restent \`a entrer.  Ainsi, m\^eme si la liste des coupons est trop 
longue pour tenir sur l'\'ecran, au moins le bas de la liste sera 
lisible.  Une autre solution moins commode consiste \`a revenir au 
menu principal, puis \`a choisir l'option `\verb|V|' pour voir les 
appariements et les r\'esultats partiels.

	PAPP vous signalera quand les r\'esultats seront complets.
Relisez-vous (\'eventuellement en utilisant [Tab] pour vous rappeler des
coupons entr\'es quelque temps avant), et si tout semble correct, tapez
simplement un `\verb|!|' en r\'eponse au prompt, puis [Return]; apr\`es
avoir enregistr\'e les r\'esultats, PAPP vous affichera le nouveau
classement. 

	A des fins de test, j'ai ajout\'e dans le m\^eme sous-menu une
option `\verb|?|' pour tirer al\'eatoirement les r\'esultats des rondes
--- plus pr\'ecisement, de celles dont les r\'esultats n'ont pas encore
\'et\'e entr\'es.  C'est surtout int\'eressant quand PAPP conna{\^\i}t
le classement des joueurs, car il peut estimer raisonnablement les
probabilit\'es relatives de gain.

\subsection{Chargement et sauvegarde}

	Il n'y a pas d'option ``Charger'' ou ``Sauver''.  Les
sauvegardes sont faites quand c'est n\'ecessaire (voir le chapitre
``Mode de sauvegarde'' pour plus de d\'etails) et lorsqu'on quitte le
programme.  Quand on relance PAPP, celui-ci relit le fichier
interm\'ediaire pr\'ec\'edemment cr\'e\'e sur le disque (celui-ci
aura pour nom \verb|__papp__|, typiquement) et se remet exactement
dans l'\'etat o\`u il \'etait la derni\`ere fois qu'on l'a quitt\'e.

	Ceci a une cons\'equence importante: avant un nouveau tournoi,
il faut penser \`a effacer ce fichier du disque, sinon PAPP se remettrait
dans l'\'etat ou il \'etait apr\`es la derni\`ere ronde du tournoi
pr\'ec\'edent, ce qui n'est g\'en\'eralement pas l'effet recherch\'e.

\subsection{Appariements manuels}

Il est possible de forcer manuellement les appariements avec 
`\verb|M|'.  Pour des raisons de place disponible sur l'\'ecran, il ne 
m'a pas sembl\'e possible de donner les noms complets des joueurs ``en 
clair''.  Les joueurs sont donc repr\'esent\'es par les huit premiers 
caract\`eres de leur nom, suivi de leur num\'ero (entre 
parenth\`eses); heureusement, il est possible d'afficher la liste ``en 
clair'' des joueurs et des appariements sans revenir au menu 
principal, puisque les options `\verb|L|', `\verb|V|' et `\verb|F|' 
sont \'egalement disponibles dans ce sous-menu.

	PAPP affiche d'abord les couples de joueurs appari\'es (comme
toujours, le premier joueur a les noirs), puis les joueurs non
appari\'es.  Ces deux listes sont mises \`a jour apr\`es chaque
op\'eration.  Il y a trois op\'erations possibles:

\begin{itemize}

	\item[`{\tt Z}':] d\'etruit tous les appariements. Cette commande ne
prend aucun argument, mais PAPP vous demandera confirmation.

\item[`{\tt D}':] d\'ecoupler un joueur. Cette commande prend un 
argument, qui est le nom ou le num\'ero du joueur.  Si ce joueur 
\'etait appari\'e, son appariement est d\'etruit.  Sinon, la commande 
est sans effet.  Comme toujours lorsque PAPP vous demande un nom de 
joueur, vous pouvez entrer seulement les premi\`eres lettres de son 
nom, ou obtenir une compl\'etion du tampon d'entr\'ee en appuyant sur 
la touche [Tab].

\item[`{\tt A}':] apparier deux joueurs.  PAPP vous demande 
successivement le nom (ou le num\'ero) du joueur noir, puis du joueur 
blanc.  Si ces joueurs \'etaient d\'ej\`a appari\'es, leurs 
pr\'ec\'edents appariements sont d\'etruits.

\end{itemize}

\subsection{Lecture des r\'esultats}

	La commande `\verb|L|' donne le score et le d\'epartage des
joueurs \`a un moment donn\'e du tournoi; les donn\'ees affich\'ees ne
seront modifi\'ees qu'apr\`es que les r\'esultats de la ronde auront
\'et\'e enti\`erement entr\'es et valid\'es.  Le format d'affichage est
le suivant: le rang dans le tournoi, le score puis le d\'epartage entre
crochets, un \'eventuel `\verb|-|' pour indiquer que le joueur est
absent (i.e., a abandonn\'e le tournoi), puis le nom, le num\'ero et la
nationalit\'e du joueur.

Il est possible d'obtenir les r\'esultats individuels d'un joueur avec 
la commande `\verb|F|' du menu principal. PAPP affiche alors les 
appariements et les r\'esultats des rondes pr\'ec\'edentes, sous la 
forme suivante: le num\'ero de ronde, la couleur du joueur, puis le 
score de sa partie (eventuellement pr\'ec\'ed\'e d'un `\verb|+|' pour 
indiquer une victoire ou un `\verb|=|' en cas de nulle), et enfin le 
nom et le num\'ero Elo de son adversaire.  Si le fichier des joueurs 
contient des renseignements suppl\'ementaires sur le joueur, ils sont 
affich\'es tels quels en bas de la fiche.  Pour PAPP, ce ne sont que 
des commentaires: aucune interpr\'etation n'est faite.  Ces 
commentaires sont \'egalement affich\'es lors de l'inscription du 
joueur.

	Si vous avez entr\'e seulement une partie des r\'esultats d'une
ronde, vous pouvez v\'erifier ce que vous avez d\'ej\`a entr\'e avec la
commande `\verb|V|' du menu principal (encore une fois, le premier nomm\'e
a les noirs).

\subsection{Autres commandes}


La commande `\verb|E|' du menu principal produit un fichier Elo qui 
peut \^etre relu par JECH. A n'utiliser qu'\`a la fin du tournoi.

La commande `\verb|C|' permet de corriger un r\'esultat erron\'e dans 
une ronde pass\'ee.  Pour plus de d\'etails, voyez le chapitre 
``Correction d'un r\'esultat'' plus loin dans ce manuel.

La commande `\verb|S|' permet de sauvegarder le classement dans un 
fichier.  Si le nom de fichier commence par un `\verb.|.'  (``pipe''), 
celui-ci est interpr\'et\'e comme une redirection vers une commande 
(cette facilit\'e n'est pas disponible sur tous les syst\`emes 
d'exploitation).  De m\^eme, les commandes [Ctrl-V] et [Ctrl-F] 
permettent de rediriger vers un fichier (ou un pipe) la liste des 
appariements et la fiche d'un joueur, respectivement.  Si vous 
d\'esirez une sauvegarde automatique, ronde par ronde, des r\'esultats 
des parties et des classements interm\'ediaires, voyez le chapitre 
``R\'ecapitulatif ronde par ronde'' plus loin dans ce manuel.

	La commande `\verb|X|' quitte le programme, en faisant toutes les
sauvegardes n\'ecessaires.

	La commande [Ctrl-Z] suspend le programme, si le syst\`eme
d'exploitation conna{\^\i}t le ``job control''.

	La commande `\verb|&|' affiche un petit message de copyright, le
num\'ero de version, la date de compilation, et diverses limitations
statiques du programme.

	Les commandes `\verb|-|' et `\verb|+|' sont utilis\'ees quand un
joueur quitte le tournoi, ou revient dans le tournoi apr\`es l'avoir
quitt\'e.  Il est possible de faire sortir ou revenir tous les joueurs
en tapant `\verb|*|' en r\'eponse au prompt.  Enfin, la commande
`\verb|N|' permet de modifier la nationalit\'e d'un joueur.

\section{Appariements automatiques}

C'est ce qui se cache derri\`ere la commande `\verb|A|' du menu 
principal.  Tout d'abord, insistons sur le fait que `\verb|A|' ne fait 
que {\em compl\'eter\/} les appariements d\'ej\`a existants; en aucun 
cas cette commande ne peut modifier les appariements pr\'ec\'edemment 
choisis, que ce soit par l'utilisateur ou par PAPP lui-m\^eme.  En 
particulier, si les appariements sont d\'ej\`a complets, alors 
`\verb|A|' n'a pas d'autre effet que d'afficher, et \'eventuellement 
imprimer, la liste des appariements.

Les appariements sont toujours calcul\'es par optimisation (m\^eme \`a 
la premi\`ere ronde).  Tout d'abord, si le nombre de joueurs est 
impair, on ajoute un joueur fictif (Bip); on peut donc supposer qu'il 
y a $2n$ joueurs.  On d\'etermine ensuite, pour chaque couple $(i,j)$ 
de joueurs, une {\em p\'enalit\'e}, c'est-\`a-dire un nombre r\'eel 
positif $p_{ij}$ exprimant dans quelle mesure il n'est pas souhaitable 
que le joueur $i$ joue (avec les noirs) contre le joueur $j$.  Puis, 
on cherche l'appariement $(i_1,j_1)$, $(i_2,j_2)$, \dots, $(i_n,j_n)$, 
minimisant la p\'enalit\'e totale
 $$ {\it Penalite\_totale} = \sum_{k=1}^n p_{i_k j_k} $$

	Ceci est un probl\`eme classique de th\'eorie des graphes
(recherche d'un couplage de poids maximum dans un graphe) qui peut
\^etre r\'esolu en temps polynomial, typiquement en $O(n^3)$.
L'impl\'ementation actuelle de PAPP semble plut\^ot en $O(n^4)$.

\subsection{Description des p\'enalit\'es}

Les p\'enalit\'es sont calcul\'ees comme suit:
 $$\matrix{
    p_{ij} &= {\it pen\_couleur}(i,{\rm NOIR})
	    + {\it pen\_couleur}(j,{\rm BLANC})		\hfill\cr
	   &\quad{}+ {\it pen\_flottement}(i,j)		\hfill\cr
	   &\quad{}+ {\it pen\_repetition}(i,j)		\hfill\cr
	   &\quad{}+ {\it pen\_chauvinisme}(i,j)	\hfill\cr
	   &\quad{}+ {\it pen\_elitisme}(i,j)	    \hfill\cr
 }$$


\subsubsection{P\'enalit\'es de couleur}

	Le terme ${\it pen\_couleur}(i,{\it couleur})$ indique qu'il
n'est pas souhaitable que le joueur $i$ joue la couleur $\it couleur$
lors de cette ronde. Notons $\delta_i$ l'{\em \'ecart chromatique\/} de ce
joueur, c'est-\`a-dire le nombre de fois o\`u il a eu les Noirs moins le
nombre de fois ou il a eu les Blancs. Selon qu'il jouera Noir ou Blanc
\`a cette ronde, son \'ecart chromatique deviendra $\delta_i+1$ ou
$\delta_i-1$, respectivement.
On prend donc
 $$ {\it pen\_couleur}(i,{\it couleur}) =
    {\bf p\_coul}\Bigl[\left|\delta_i\pm1\right|\Bigr]
  + {\it pen\_couleur\_repetee} $$
Le terme $\it pen\_couleur\_repetee$ vaut $\bf p\_repcl$ si le joueur
avait d\'ej\`a jou\'e cette couleur \`a la ronde pr\'ec\'edente, et $0$
sinon.  On a toujours ${\bf p\_coul}[0] = 0$.

\subsubsection{P\'enalit\'es de flottement}	
	Le terme ${\it pen\_flottement}(i,j)$ dit qu'il n'est pas
souhaitable de faire jouer entre eux deux joueurs dont les scores sont
trop diff\'erents.  On prend ici
 $${\it pen\_flottement}(i,j) =
    {\bf p\_flot}[f] + {\it corr\_flot}(i) + {\it corr\_flot}(j)
 $$
o\`u $f=\left| S_i-S_j \right|$ est le flottement entre les deux
joueurs, exprim\'e en demi-points.  On a toujours ${\bf p\_flot}[0]=0$.
Le terme correctif ${\it corr\_flot}(i)$ vaut $+{\bf p\_flcum}$ si le
joueur flotte alors qu'il avait d\'ej\`a flott\'e dans le m\^eme sens
\`a la ronde pr\'ec\'edente.  Il vaut $-{\bf min\_fac}$ si le joueur
flotte dans l'autre sens.  Il vaut z\'ero si le joueur n'a pas flott\'e
\`a cette ronde ou \`a la ronde pr\'ec\'edente, ou s'il s'agit de la
premi\`ere ronde.  Un joueur qui joue contre Bip flotte toujours bas
(pour savoir de combien, voir plus bas). Les corrections, si elles sont
n\'egatives, ne peuvent exc\'eder en valeur absolue le terme qu'elles
sont suppos\'ees corriger, si bien que ${\it pen\_flottement}(i,j)$ sera
toujours positif (voir la section suivante).


\subsubsection{P\'enalit\'es de r\'ep\'etition} 	
	Le terme ${\it pen\_repetition}(i,j)$ emp\^eche de faire jouer
entre eux deux joueurs qui se sont d\'ej\`a rencontr\'es lors d'une
ronde pr\'ec\'edente.  Si $i$ et $j$ ont d\'ej\`a jou\'e ensemble avec
les m\^emes couleurs, on ajoute la p\'enalit\'e $\bf p\_mcoul$; s'ils
ont d\'ej\`a jou\'e ensemble avec les couleurs oppos\'ees, on ajoute
$\bf p\_clopp$.  S'ils ont d\'ej\`a jou\'e plusieurs parties ensemble,
les p\'enalit\'es ci-dessus s'ajoutent.  Enfin, on ajoute la
p\'enalit\'e $\bf p\_desuite$ si les deux joueurs se sont rencontr\'es
\`a la ronde imm\'ediatement pr\'ec\'edente. 


\subsubsection{P\'enalit\'es de chauvinisme}  	
	Le terme ${\it pen\_chauvinisme}(i,j)$ p\'enalise les parties
entre joueurs d'un m\^eme pays: il vaut ${\bf p\_chauv}[{\it
numero\_ronde}]$ si $i$ et $j$ sont de m\^eme nationalit\'e, et $0$
sinon.


\subsubsection{P\'enalit\'es d'\'elitisme} 
	Le terme ${\it pen\_elitisme}(i,j)$ dit que, quand il est n\'ecessaire 
d'avoir des matchs entre deux joueurs dont les scores sont 
diff\'erents, il est pr\'ef\'erable de r\'epartir ces matchs 
d\'es\'equilibr\'es dans le bas du classement plut\^ot que dans le 
haut du classement.  On prend ici 
$${\it pen\_elitisme}(i,j) = 
({\bf p\_elit}[{\it numero\_ronde}] \fois (S_i+S_j) \fois f ) / 2
 $$
o\`u $f=\left| S_i-S_j \right|$ est le flottement entre les deux
joueurs, exprim\'e en demi-points.



\subsubsection{Le cas de Bip} 

	Si l'un des joueurs est Bip, les p\'enalit\'es se simplifient un
peu, parce qu'il n'y a pas \`a se soucier de la couleur. On prend
 $$\matrix{
   p_{i,{\rm Bip}} = p_{{\rm Bip},i} &=
   	{\it pen\_couleur}(i, {\rm GRIS})		\hfill\cr
	&\quad{}+{\it pen\_flottement}(i,{\rm Bip})	\hfill\cr
	&\quad{}+{\it pen\_repetition}(i,{\rm Bip})	\hfill\cr
	&\quad{}+{\it pen\_elitisme}(i,{\rm Bip})	\hfill\cr
 }
 $$
 
 \begin{itemize}
 
 \item
Comme l'\'ecart chromatique n'est pas modifi\'e quand on joue contre
Bip, la p\'enalit\'e de couleur vaut
$$
  {\it pen\_couleur}(i,{\rm GRIS}) =
    {\bf p\_coul}\Bigl[\left| \delta \right|\Bigr]
$$
 
 \item
	Les p\'enalit\'es $\it pen\_flottement$ et $\it pen\_elitisme$ 
	sont d\'efinies comme plus haut; cependant, bien que le score de 
	Bip soit de z\'ero point, on fait ``comme si'' son score \'etait 
	imm\'ediatement inf\'erieur, d'un demi-point, \`a celui du dernier 
	joueur pr\'esent (ceci pour \'eviter d'avoir des p\'enalit\'es de 
	flottement vraiment excessives).  Donc, en notant $f = S_i - 
	S_{\rm min} + 1$:
	
	
$$ 
  {\it pen\_flottement}(i,{\rm Bip}) =
    {\bf p\_flot}[f] 
$$
$$
  {\it pen\_elitisme}(i,{\rm Bip}) =
    ({\bf p\_elit}[{\it numero\_ronde}] \fois (S_i+S_{\rm min} - 1) 
     \fois f) / 2 
$$ 
 \item
	Dans le terme $\it pen\_repetition$, la distinction entre $\bf
p\_mcoul$ et $\bf p\_clopp$ dispara{\^\i}t, il n'y a plus qu'une seule
p\'enalit\'e \`a consid\'erer: on ajoute $\bf p\_bipbip$ autant de fois
que le joueur a jou\'e contre Bip aux rondes pr\'ec\'edentes, et comme
plus haut on ajoute $\bf p\_desuite$ si le joueur a jou\'e contre Bip \`a
la ronde imm\'ediatement pr\'ec\'edente. 

\end{itemize}

\subsection{Conditions sur les p\'enalit\'es}

	PAPP n'impose qu'un minimum de conditions de coh\'erence sur les
p\'enalit\'es; si ces conditions ne sont pas remplies, le programme
se termine en signalant une erreur fatale. Ces conditions sont les
suivantes:

\begin{itemize}

\item	Les p\'enalit\'es de couleur doivent cro{\^\i}tre avec
l'\'ecart chromatique:
$$ 0 = {\bf p\_coul}[0] \le {\bf p\_coul}[1] \le {\bf p\_coul}[2] \le\cdots$$

\item	Les p\'enalit\'es de flottement doivent cro{\^\i}tre avec
le flottement; et

\item	la quantit\'e ${\bf min\_fac}$ doit \^etre inf\'erieure ou
\'egale \`a la moiti\'e de la plus petite p\'enalit\'e de flottement:
$$ 0 = {\bf p\_flot}[0] \le 2\fois{\bf min\_fac} \le {\bf p\_flot}[1] \le
	{\bf p\_flot}[2] \le \cdots $$

\end{itemize}

	Ces conditions ne suffisent \'evidemment pas \`a garantir des
appariements ``raisonnables''; il faudrait au moins s'assurer que les
penalites de couleur et de flottement croissent ``suffisamment vite'' en
fonction de l'\'ecart chromatique ou du flottement. La troisi\`eme
condition garantit que les p\'enalites de flottement apr\`es correction
sont toujours positives (voir la section pr\'ec\'edente).

\subsection{Derni\`ere passe}

Une fois l'optimisation r\'ealis\'ee, PAPP examine s'il existe
d'autres appariements optimaux, se d\'eduisant du premier uniquement
par des interversions de couleurs.  Quand c'est le cas ---
c'est-\`a-dire pour toute paire $\{i,j\}$ de joueurs appari\'es telle
que $p_{ij}=p_{ji}$ --- PAPP choisira les couleurs comme suit: on
cherche la ronde la plus r\'ecente o\`u $i$ et $j$ avaient des couleurs
diff\'erentes, et on inverse ces couleurs.  Si ce n'est pas possible,
c'est-\`a-dire si les deux joueurs ont toujours jou\'e avec les
m\^emes couleurs, alors on tire les couleurs au sort.

Insistons sur le fait que cette derni\`ere passe ne d\'egrade 
nullement la qualit\'e de l'appa\-rie\-ment; PAPP choisit simplement entre plusieurs 
appariements optimaux.

\section{Le fichier de configuration}

% Macros pour la grammaire
% Laisser les espaces : ils sont transparents en TeX, 
% mais utiles en HTML.
%
\def\gram#1{$\langle\hbox{\it#1}\rangle$}  
\def\opt#1{$[\hbox{\it#1}]$}     % TtH ne connait pas \lbrack et \rbrack !
\def\optr#1{$\lbrace\hbox{\it#1}\rbrace$}
\def\fleche{$  \longrightarrow\,\,\,  $}
\def\ou{$\vert\,\tthdump{\,}$}
\def\vrb#1{{\tt #1}}	% le verbatim du pauvre!

	Ce fichier est charg\'e par PAPP d\`es l'ex\'ecution.  Le nom
de ce fichier est d\'etermin\'e \`a partir de la ligne de commande, si
celle-ci contient exactement un argument.  Sinon, PAPP utilise la
variable d'environnement \verb|PAPP_CFG|, si celle-ci est d\'efinie;
sinon, PAPP prendra \verb|papp.cfg| comme nom de fichier (dans le
r\'epertoire courant).

	Si PAPP ne peut ouvrir le fichier de configuration, il
cr\'eera un fichier \verb|papp.cfg| dans le r\'epertoire courant,
contenant ses r\'eglages par d\'efaut.
 
	Ce fichier de configuration contient les noms des autres
fichiers requis par PAPP, la liste des p\'enalit\'es, et quelques
autres param\'etrages.
 
\subsection{Conventions lexicales}

	Le texte est d\'ecoup\'e en cha{\^\i}nes, entiers, mots-cl\'es
et caract\`eres isol\'es.  Les espaces, tabulations, retours chariot
et passages \`a la ligne sont ignor\'es; c'est le point-virgule qui
tient lieu de fin de commande (voir plus bas).

	Une \gram{chaine} est une cha{\^\i}ne de caract\`eres
d\'elimit\'ee par des doubles guillemets (\verb|"|), tenant sur une
seule ligne.

	Un mot-cl\'e est une s\'equence r\'eserv\'ee de lettres et de
tirets (`\verb|-|'), le tiret ne pouvant pas appara{\^\i}tre comme
premier caract\`ere.  Aucune distinction n'est faite entre majuscules
et minuscules.

	Un \gram{entier} d\'esigne un nombre entre $0$ et $2^{31}-1 =
2\,147\,483\,647$; ce dernier nombre peut \'egalement \^etre obtenu
avec le mot-cl\'e \verb|INFINI| --- comme tous les mots-cl\'es, on
pourrait aussi l'\'ecrire en minuscules, mais on n'aurait plus
l'impression qu'il s'agit d'un grand nombre.  (Note: cette limite
d\'epend en r\'ealit\'e de l'impl\'ementation; un \gram{entier} \'etant
repr\'esent\'e en C par un \verb|signed long|, la limite serait
$2^{63}-1$ sur les machines 64-bits telles que le DEC Alpha.)

	Si un `\verb|%|' ou `\verb|#|' appara{\^\i}t sur une ligne, tout
le reste de la ligne est ignor\'e (ainsi que le caract\`ere lui-m\^eme),
sauf si le `\verb|%|' est imm\'ediatement suivi par un `\verb|_|'; la
s\'equence `\verb|%_|' est ignor\'ee (elle \'equivaut \`a un espace),
donc le reste de la ligne n'est {\em pas\/} ignor\'e.

	Si le mot \verb|__eof__| appara{\^\i}t dans le fichier, tout
ce qui suit est ignor\'e (le fichier est imm\'ediatement referm\'e).

\subsection{Grammaire}

La syntaxe du fichier de configuration est la suivante:

\medbreak

\halign{\noindent\hskip 1cm\relax#\hfil\cr
    \gram{fichier-config} \fleche \optr{\gram{commande}  \vrb;}			\cr
    \gram{commande} \fleche \gram{vide}						\cr
	\qquad \ou \vrb{fichier} (\vrb{joueurs} \ou \vrb{nouveaux} \ou \vrb{inter}) 
	\vrb{=} \gram{chaine}	\cr
	\qquad \ou \vrb{fichier} (\vrb{appariements} \ou \vrb{resultats} \ou \vrb{classement}) 
	\vrb{=} \gram{chaine}	\cr
	\qquad \ou \vrb{pays =} \gram{chaine}				\cr
	\qquad \ou \vrb{brightwell =} \gram{entier}				\cr
	\qquad \ou \vrb{score-bip =} \gram{entier} \opt{\vrb/ \gram{entier}}	\cr
	\qquad \ou \vrb{sauvegarde} (\vrb{immediate} \ou \vrb{differee})	\cr
	\qquad \ou \vrb{impression} (\vrb{manuelle} \ou \vrb{automatique} \gram{entier})	\cr
	\qquad \ou \vrb{affichage-pions} (\vrb{absolu} \ou \vrb{relatif})	\cr
	\qquad \ou \vrb{couleurs = \char`\{} \gram{chaine} \vrb, 
	 \gram{chaine} \vrb{\char`\}}	\cr
	\qquad \ou \vrb{zone-insertion} \opt{chaine}
		\vrb= \gram{entier} \vrb- \gram{entier}				\cr
	\qquad \ou \vrb{toutes-rondes} \vrb= \gram{entier}
		\vrb- \gram{entier} \vrb{joueurs}				\cr
	\qquad \ou \gram{indication-penalites} \ou \gram{commande-interne}   	\cr
%    \gram{type-fichier} \fleche \vrb{joueurs} \ou \vrb{nouveaux} \ou \vrb{inter}	\cr
%    \gram{type-sauvegarde} \fleche \vrb{immediate} \ou \vrb{differee}		\cr
}

\medbreak

\subsection{Liste des p\'enalit\'es}

La liste des p\'enalit\'es se d\'ecompose en cinq sections: les 
p\'enalit\'es de couleur, celles de flottement, celles de 
r\'ep\'etition, celles de chauvinisme et celles d'\'elitisme.  Chaque 
section commence par un label indiquant de quelle section il s'agit, 
puis la liste des p\'enalit\'es de cette section.

\medbreak
\halign{\noindent\hskip 1cm\relax#\hfil\cr
    \gram{indication-penalites} \fleche \vrb{penalites \char`\{} \optr{section}
     \vrb{\char`\}}	\cr
    \gram{section} \fleche \gram{section-couleur} \ou \gram{section-flottement} \cr
	\qquad \qquad \qquad  \ou \gram{section-repetition} \ou \gram{section-chauvinisme} \cr
	\qquad \qquad \qquad  \ou \gram{section-elitisme} \cr
}
\medbreak

\subsubsection{P\'enalit\'es de couleur}

La section ``couleur'' permet de choisir les penalit\'es ${\bf p\_coul}
[\delta]$ et $\bf p\_repcl$. La syntaxe est

\medbreak
\halign{\noindent\hskip 1cm\relax#\hfil\cr
    \gram{section-couleur} \fleche \vrb{Couleur :}
		\optr{\gram{penalite-couleur} \vrb;}	\cr
    \gram{penalite-couleur} \fleche \gram{entier} \opt{\vrb+}
		\vrb{fois =} \gram{pen}			\cr
	\qquad \qquad \qquad \qquad \qquad \ou \vrb{de-suite =} \gram{pen}		\cr
    \gram{pen} \fleche un \gram{entier} entre $0$ et $10\,000\,000$	\cr
}
\medbreak

\noindent Consid\'erons l'exemple suivant:
\begin{verbatim}
    Couleur:
        2  fois  =  500;  % delta == 2
        3+ fois  = 5000;  % delta >= 3
        de-suite =  100;  % p_repcl
\end{verbatim}
Cela signifie qu'on aura une penalit\'e de $500$ points si l'\'ecart de
couleur ($\delta$) vaut deux, et $5000$ s'il vaut trois {\em ou plus\/}; et
il y a $100$ points de p\'enalit\'e si l'on joue deux fois de suite avec la
m\^eme couleur.

\subsubsection{P\'enalit\'es de flottement}

La section ``flottement'' concerne les variables ${\bf p\_flot}(f)$,
$\bf p\_flcum$ et $\bf min\_fac$. La syntaxe est

\medbreak
\halign{\noindent\hskip 1cm\relax#\hfil\cr
    \gram{section-flottement} \fleche \vrb{Flottement :}
		\optr{\gram{penalite-flottement} \vrb;}	\cr
    \gram{penalite-flottement} \fleche \gram{entier} \opt{\vrb+}
		\vrb{demi-point =} \gram{pen}		\cr
	\qquad \qquad \qquad \qquad \qquad \ou \vrb{de-suite =} \gram{pen}		\cr
	\qquad \qquad \qquad \qquad \qquad \ou \vrb{minoration =} \gram{pen}		\cr
}
\medbreak

\noindent On notera que \verb|demi-point| peut prendre un `s' au pluriel.
Voici un exemple:
\begin{verbatim}
    Flottement:
        1  demi-point  =  100;  % p_flot[1]
        2  demi-points =  500;  % p_flot[2]
        3+ demi-points = 5000;  % p_flot[3 et plus]
        de-suite       =   50;  % p_flcum
        minoration     =   10;  % min_fac
\end{verbatim}

\subsubsection{P\'enalit\'es de r\'ep\'etition}

Ensuite vient la section ``r\'ep\'etition'', qui concerne les variables
$\bf p\_mcoul$, $\bf p\_clopp$, $\bf p\_bipbip$ et $\bf p\_desuite$. Ces
variables sont initialis\'ees par les d\'eclara\-tions suivantes,
respectivement:

\medbreak
\halign{\noindent\hskip 1cm\relax#\hfil\cr
    \gram{section-repetition} \fleche \vrb{Repetition :}
    		\optr{\gram{penalite-repetition} \vrb;}		\cr
    \gram{penalite-repetition} \fleche \vrb{memes-couleurs =} \gram{pen}	\cr
	\qquad \qquad \qquad \qquad \qquad \ou \vrb{couleurs-opposees =} \gram{pen}		\cr
	\qquad \qquad \qquad \qquad \qquad \ou \vrb{bip-bip =} \gram{pen}			\cr
	\qquad \qquad \qquad \qquad \qquad \ou \vrb{de-suite =} \gram{pen}			\cr
}
\medbreak

\subsubsection{P\'enalit\'es de chauvinisme}

Le tableau ${\bf p\_chauv}[r]$ est initialis\'e par la section
``chauvinisme'', de la mani\`ere suivante:

\medbreak
\halign{\noindent\hskip 1cm\relax#\hfil\cr
    \gram{section-chauvinisme} \fleche \vrb{Chauvinisme :}
    		\optr{\gram{penalite-chauvinisme} \vrb;}		\cr
    \gram{penalite-chauvinisme} \fleche \vrb{ronde} \gram{entier} \opt{\vrb+}
    	\vrb= \gram{pen}	\cr
}
\medbreak

Les rondes sont num\'erot\'ees \`a partir de un. Si par exemple on veut
que la p\'enalit\'e de chauvinisme soit \'egale \`a $100$ pour les dix
permi\`eres rondes et \`a $1000$ ensuite, on \'ecrira:
\begin{verbatim}
    Chauvinisme:
        ronde  1+ =  100;  % ronde 1  et suivantes
        ronde 11+ = 1000;  % ronde 11 et suivantes
\end{verbatim}

\subsubsection{P\'enalit\'es d'\'elitisme}

Le tableau ${\bf p\_elit}[r]$ est initialis\'e par la section
``\'elitisme'', de la mani\`ere suivante:

\medbreak
\halign{\noindent\hskip 1cm\relax#\hfil\cr
    \gram{section-elitisme} \fleche \vrb{Elitisme :}
    		\optr{\gram{penalite-elitisme} \vrb;}		\cr
    \gram{penalite-elitisme} \fleche \vrb{ronde} \gram{entier} \opt{\vrb+}
    	\vrb= \gram{pen}	\cr
}
\medbreak


Les rondes sont num\'erot\'ees \`a partir de un.  Si par exemple on 
veut que la p\'enalit\'e d'\'elitisme soit \'egale \`a $5$ pour les 
cinq permi\`eres rondes, \`a $25$ pour les rondes six \`a dix, et \`a 
$100$ ensuite, on \'ecrira:
\begin{verbatim}
    Elitisme:
        ronde  1+ =   5;  % ronde 1  et suivantes
        ronde  6+ =  25;  % ronde 6  et suivantes
        ronde 11+ = 100;  % ronde 11 et suivantes
\end{verbatim}

\subsection{Remarques diverses}

	D\`es que PAPP rencontre le mot-cl\'e \verb|penalites|, il remet
\`a z\'ero {\em toutes\/} les p\'enalit\'es. Ceci permet de partir sur
des bases saines, en \'evitant que les p\'enalit\'es d\'efinies par
l'utilisateur ne viennent se m\'elanger avec celles par d\'efaut. En
particulier, si l'utilisateur omet de d\'eclarer une p\'enalit\'e,
celle-ci restera \'egale \`a z\'ero. Ce syst\`eme garantit \'egalement une
certaine compatibilit\'e ascendante entre les diff\'erentes versions de
PAPP.

	Pour des raisons historiques, les p\'enalit\'es ne peuvent
exc\'eder 10 millions; les p\'enalit\'es plus grandes seront ramen\'ees
\`a cette valeur (avec un message d'avertissement).

\subsection{Tournois toutes-rondes}

	Bien qu'initialement pr\'evu pour organiser des tournois selon
le syst\`eme suis\-se (ou l'une de ses multiples variantes), PAPP peut
\'egalement organiser des tournois toutes rondes ou plusieurs fois
toutes rondes.  Une directive de la forme
 $$\hbox{\verb|toutes-rondes = 1-12 joueurs;|}$$
indique \`a PAPP qu'il doit organiser un toutes-rondes s'il y a entre 1
et 12 joueurs.  Si ce n'est pas possible, c'est-\`a-dire s'il y a trop
de joueurs, ou bien si des joueurs ont quitt\'e le tournoi ou sont
entr\'es apr\`es la fin de la premi\`ere ronde, ou encore si vous avez
forc\'e certains appariements \`a la main, alors PAPP se repliera sur
l'algorithme d'optimisation d\'ecrit plus haut. 

	L'arbitre doit s'assurer que le nombre de rondes est un multiple
du nombre de joueurs moins un, Bip compris.

\subsection{Le d\'epartage de Brightwell}

	Par d\'efaut, le d\'epartage se fait au nombre de pions.  Ce
sera \'egalement le cas si votre fichier de configuration contient la
directive \verb|brightwell = 0|.  

Si au contraire vous sp\'ecifiez une valeur strictement positive,
celle-ci sera prise comme valeur du coefficient $\beta$ de Brightwell. 
Rappelons que le d\'epartage de Brightwell est donn\'e par
 $${\it departage} = {\it nombre\_pions} + \beta\fois{\it buchholz}$$
Dans cette formule, le $\it buchholz$ est la somme des scores, en 
demi-points, des adversaires rencontr\'es. Une partie contre Bip, ou 
une partie contre un adversaire qui a abandonn\'e le tournoi, est 
remplac\'ee par une partie nulle contre soi-m\^eme, avant d'appliquer 
la formule ci-dessus.

Imaginons par exemple un joueur $A$, qui joue et gagne 33/31 contre 
$B$ \`a la premi\`ere ronde, et se retrouve contre Bip \`a la 
deuxi\`eme ronde.  On suppose que $B$ a gagn\'e sa deuxi\`eme partie.  
Si on prend comme coefficient de Brightwell $\beta=3$, quel est le 
d\'epartage de $A$ \`a l'issue de la deuxi\`eme ronde?

On constate que $A$ a 2 points (soit 4 demi-points) et $B$, 1 point 
(soit 2 demi-points).  La partie de $A$ contre Bip doit \^etre 
remplac\'ee par une partie nulle 32/32 contre $A$.  Le d\'epartage 
vaut donc

 $${\it departage} = (33+32) + 3 \fois (2+4) = \hbox{83 pions}.$$

Insistons sur le fait que le d\'epartage de Brightwell est calcul\'e 
dans PAPP \`a partir de la somme {\it en demi-points} des score des 
adversaires rencontr\'es, tandis que les r\`eglements \'ecrits des 
tournois fixent g\'en\'eralement la valeur du coefficient de 
Brightwell pour des scores en points entiers.  Dans ce cas, la valeur 
du coefficient de Brightwell sp\'ecifi\'ee dans le fichier de 
configuration doit \^etre {\it la moiti\'e} de celle apparaissant dans 
le r\`eglement.

Notes : les commandes `\verb|-|' et `\verb|+|' ont un effet de bord 
sur le d\'epartage de Brightwell, parce que les joueurs absents (\`a 
un moment donn\'e) sont suppos\'es avoir abandonn\'e le tournoi.  
D'autre part, pour les jeux ``sans pions'' (voir ``Autres jeux''), le 
coefficient de Brightwell est toujours ignor\'e, et c'est le Buchholz 
qui sert de d\'epartage.



\section{Le fichier des joueurs}

\subsection{Syntaxe des joueurs}

	PAPP peut lire tout fichier de joueurs accept\'e par JECH. Un tel
fichier contient un joueur par ligne, ainsi que des lignes de la forme
 $$\hbox{\verb|pays =| \gram{pays-abrege}}$$
pour signaler que tous les joueurs qui vont suivre sont du pays
indiqu\'e---au moins jusqu'\`a la prochaine ligne de ce type (les
guillemets autour de \gram{pays-abrege} sont facultatifs, mais le nom de
pays ne doit pas contenir d'espace).  Les lignes d\'ecrivant un joueur
doivent sp\'ecifier le num\'ero Elo, puis le nom et le(s) pr\'enom(s)
(\'eventuellement s\'epar\'es par une virgule s'il y a
ambigu{\"\i}t\'e); on peut ensuite pr\'eciser le pays abr\'eg\'e, entre
accolades, puis le classement, entre chevrons, et un commentaire apr\`es
un accent grave; chacun de ces trois derniers champs est optionnel. 
Exemple:
\begin{verbatim}
   92 FELDBORG, Karsten {DK} <2209> `Vive les Danois!
\end{verbatim}
est accept\'e par PAPP.  Mais pas par JECH, malheureusement, qui ne
comprend pas les signes \verb|<| et \verb|>|, ni l'accent grave; or on
aimerait bien que le m\^eme fichier de joueurs soit partag\'e par JECH
et PAPP.  L'astuce consiste a r\'e\'ecrire la ligne pr\'ec\'edente comme
suit:
\begin{verbatim}
   92 FELDBORG, Karsten {DK} %_<2209> `Vive les Danois!
\end{verbatim}

	L'id\'ee est que JECH ignore tout ce qui suit le `\verb|%|'; alors
que pour PAPP, la s\'equence `\verb|%_|' est ``transparente'', et tout se
passe comme s'il y avait un espace \`a la place. 

	Notez par ailleurs que JECH et PAPP cessent de lire le fichier
d\`es qu'ils rencontrent \verb|__eof__|; cette astuce peut \^etre
combin\'ee avec la pr\'ec\'edente pour sauter des parties enti\`eres du
fichier. 

	Le fichier fourni dans la distribution du programme se conforme
\`a cette convention, et donc peut \^etre utilis\'e \`a la fois par JECH
et PAPP.  Par d\'efaut, PAPP cherchera le fichier \verb|joueurs| dans le
r\'epertoire courant, mais il est possible de lui faire lire un autre
fichier en pla\c cant dans votre fichier de configuration
 $$\hbox{\verb|fichier joueurs =| \gram{nom-fichier} \verb|;|}$$

	Les commentaires peuvent, par exemple, \^etre utilis\'es pour
savoir si un joueur est adh\'erent et \`a jour de cotisation.  Ils sont
affich\'es lors de l'inscription des joueurs, et peuvent \^etre relus
\`a tout moment avec la commande \verb|F| du menu principal. 

\section{Autres options}

\subsection{Les zones d'insertion}

	Ce sont de grandes plages de num\'eros Elo non encore (tous)
attribu\'es, destin\'ees \`a l'inscription des nouveaux joueurs.  Il est
possible de d\'efinir une ou plusieurs zones d'insertion pour chaque
pays, et une ou plusieurs zones ``internationales'', o\`u quiconque
pourra s'inscrire.
Supposons par exemple que le fichier de configuration contienne
\begin{verbatim}
    zone-insertion "F"  =  700- 900;  % francais
    zone-insertion "GB" = 1500-1700;  % anglais
    zone-insertion      = 5000-6000;  % tous les autres
\end{verbatim}

	Les nouveaux joueurs fran{\c c}ais obtiendront un num\'ero entre
700 et 900, les joueurs anglais un num\'ero entre 1500 et 1700, et les
joueurs de tous les autres pays un num\'ero entre 5000 et 6000.

	Pour choisir un num\'ero, PAPP commence par scruter toutes les
zones d'insertion correspondant \`a la nationalit\'e du joueur. S'il n'y
en a pas ou si elles sont toutes pleines, PAPP cherchera un emplacement
libre dans les zones internationales. Si ce n'est pas possible non plus,
PAPP choisira le plus petit num\'ero $\ge1$ non attribu\'e. Dans tous
les cas, l'arbitre est libre de choisir un autre num\'ero que celui
propos\'e par PAPP.

	Les nouveaux joueurs sont alors enregistr\'es dans un fichier
dont le nom par d\'efaut est \verb|nouveaux|, mais ceci peut \^etre
chang\'e si votre fichier de configuration contient une ligne de la
forme
 $$\hbox{\verb|fichier nouveaux =| \gram{nouveau nom} \verb|;|}$$
Ce fichier est au m\^eme format que le fichier principal des joueurs. On a
pr\'ef\'er\'e cr\'eer un autre fichier plut\^ot que de ``polluer'' le fichier
principal avec des informations moins s\^ures.

\subsection{Les appariements de la premi\`ere ronde}

	Les appariements de la premi\`ere ronde sont, pour l'essentiel,
al\'eatoires.  Comme pour toutes les autres rondes, l'appariement est
produit par optimisation, donc si la p\'enalit\'e de chauvinisme est non
nulle, PAPP aura tendance \`a apparier des joueurs de nationalit\'es
diff\'erentes.  En ce sens, l'appariement n'est pas al\'eatoire: s'il y
a plus de fran{\c c}ais que d'\'etrangers \`a un tournoi, il est garanti
que chaque \'etranger jouera contre un fran{\c c}ais; et que c'est un
fran{\c c}ais qui jouera contre Bip s'il y a un nombre impair de
joueurs. 

	En d'autres termes: la probabilit\'e que deux joueurs jouent
ensemble \`a la premi\`ere ronde ne d\'epend que de leur nationalit\'e. 
Si la p\'enalit\'e de chauvinisme de la premi\`ere ronde est nulle,
l'appariement sera rigoureusement al\'eatoire. 

\subsection{R\'ecapitulatif ronde par ronde}

Il est possible de demander \`a PAPP de sauvegarder automatiquement 
(et m\^eme d'envoyer automatiquement \`a une \'eventuelle imprimante), 
\`a la fin de chaque ronde, un r\'ecapitulatif des r\'esultats des 
parties qui viennent de se terminer, et le nouveau classement des 
joueurs dans le tournoi.

 Si le fichier de configuration contient une ligne de la
forme 
$$\hbox{\verb|fichier resultats =| \gram{nom de fichier} \verb|;|}$$
dans laquelle \gram{nom de fichier} est un nom g\'en\'erique (non 
vide) de fichier, alors PAPP, au moment o\`u vous confirmez les 
coupons, va sauvegarder les r\'esultats ``en clair'' de toutes les 
parties dans un fichier num\'erot\'e par le num\'ero de la ronde.  
Pour cr\'eer le nom du fichier pour la ronde qui vient de se terminer, 
PAPP essaye de trouver les caract\`eres `\verb|###|' dans le nom 
g\'en\'erique sp\'ecifi\'e, et les remplace par le num\'ero de la 
ronde (par exemple, si le nom g\'en\'erique dans le fichier de 
configuration est \verb|result###.txt|, PAPP cr\'eera la suite des 
fichiers \verb|result__1.txt|, \verb|result__2.txt|, 
\verb|result__3.txt|, \emph{etc.}).  Si le nom g\'en\'erique ne 
contient pas `\verb|###|', PAPP suffixera simplement ce nom 
g\'en\'erique avec le num\'ero de la ronde.

D'autre part, si le fichier de configuration contient des lignes de la
forme 
$$\hbox{\verb|fichier classement =| \gram{nom de fichier} \verb|;|}$$
$$\hbox{\verb|fichier appariements =| \gram{nom de fichier} \verb|;|}$$
dans laquelle les \gram{nom de fichier} sont des noms g\'en\'eriques (non 
vides) de fichier, alors PAPP, de la m\^eme mani\`ere, sauvegardera 
automatiquement les classements interm\'ediaires \`a la fin de chaque 
ronde, et les appariements de la ronde suivante.  Il applique les 
m\^emes conventions que ci-dessus pour construire les noms successifs 
des fichiers.

Vous pouvez tout-\`a-fait, dans le fichier de configuration, 
sp\'ecifier des \gram{nom de fichier} identiques pour les fichiers 
d'appariements, de r\'esultats ou de classement (ou pour deux d'entre 
eux) : PAPP ne cr\'eera au besoin qu'un ou deux fichier(s) par ronde, 
dans lequel il collera \`a la suite les appariements, les r\'esultats 
des parties et le classement du tournoi.  Si vous ne voulez pas 
utiliser l'une de ces trois options, vous pouvez soit commenter la 
(les) ligne(s) correspondant dans le fichier de configuration, soit 
sp\'ecifier un nom de fichier vide.  Par exemple 
\hbox{\verb|fichier resultats = ""|} annule la sauvegarde du fichier 
des r\'esultats.


Note : lors de ces sauvegardes, PAPP \'ecrase sans vergogne les 
fichiers \'eventuels qui ont le m\^eme nom que ceux qu'il est en train 
d'\'ecrire et qui \'etaient l\`a au d\'emarrage du programme.  Il faut 
donc penser \`a la fin du tournoi \`a d\'eplacer en lieu s\^ur ces 
fichiers, faute de quoi ils risqueraient d'\^etre perdus lors du 
tournoi suivant.

\subsection{Impression}

Si vous avez une imprimante allum\'ee et connect\'ee \`a votre ordinateur, la 
commande 
$$\hbox{\verb|impression automatique| \gram{nbre-de-copies} \verb|;| }$$ 
plac\'ee dans le fichier de configuration, ordonnera \`a PAPP, d\`es 
qu'il les aura cr\'e\'es, d'imprimer (avec le nombre d'exemplaires 
sp\'ecifi\'e) les fichiers d'appariements, de r\'esultats et de 
classement ronde par ronde.  Le r\'eglage par d\'efaut est le 
contraire, \verb|impression manuelle|, qui vous permet de passer dans 
un \'editeur de textes pour modifier les fichiers avant de les 
imprimer.

\subsection{Modes de sauvegarde}

Le mode de sauvegarde par d\'efaut pour le fichier interm\'ediaire 
\verb|__papp__| est ``sauvegarde diff\'er\'ee''~; ceci signifie que 
les sauvegardes sont faites seulement quand les r\'esultats d'une 
ronde ont \'et\'e enti\`erement entr\'es et valid\'es, ou bien quand 
on quitte le programme.  En particulier, les appariements ne sont pas 
sauvegard\'es, puisqu'on attend d'avoir les r\'esultats de la ronde 
pour le faire!  Ce mode est parfait pour les tests, mais dangereux en 
tournoi, parce qu'un temps consid\'erable peut s'\'ecouler entre le 
moment o\`u les appariements sont calcul\'es et celui o\`u les 
r\'esultats seront entr\'es, et on ne peut garantir que personne ne se 
sera pris les pieds dans le fil entretemps.

La commande \verb|sauvegarde immediate| ordonne \`a PAPP de faire ses 
sauve\-gardes internes aussi r\'eguli\`erement que possible, 
c'est-\`a-dire avant tout affichage et, \'evidemment, avant de 
quitter.  Plus pr\'ecisement, les commandes `\verb|L|' et `\verb|V|' 
provoquent une sauvegarde des inscrits et des appariements 
respectivement, avant tout affichage.  D'autres commandes, comme 
`\verb|A|', appellent implicitement `\verb|V|' et donc provoquent 
\'egalement une sauvegarde.  Essentiellement, si vous voyez \`a 
l'\'ecran une liste de joueurs ou d'appariements, vous pouvez 
\'eteindre la machine en toute s\'ecurit\'e parce que vous savez que 
cette liste a \'et\'e sauvegard\'ee avant d'\^etre affich\'ee.  (En 
toute rigueur, ce n'est pas enti\`erement exact: le syst\`eme 
d'exploitation peut utiliser un cache disque, et donc diff\'erer 
l'\'ecriture des donn\'ees.  Il est donc {\em toujours\/} 
pr\'ef\'erable de quitter PAPP, puis d'\'eteindre ``proprement'' la 
machine).

Une fois le tournoi termin\'e, PAPP peut cr\'eer un fichier ELO
qui sera utilis\'e pour le calcul du classement de Jech.  Choisissez
l'option `\verb|E|' du menu principal, indiquez le nom du fichier que vous
voulez cr\'eer (sous TOS ou MS-DOS, vous \^etes limit\'e a $8+3$
caract\`eres), et le nom complet du tournoi, qui peut \^etre nettement plus
long; vous avez droit \`a 127 caract\`eres, profitez-en et n'utilisez pas
d'abr\'eviations.

On notera que PAPP ne d\'etecte jamais quand le disque est
plein; et m\^eme s'il le pouvait, que pourrait-il faire?  Donc, v\'erifiez
toujours qu'il reste de la place sur vos disquettes.

\subsection{D\'epart et retour des joueurs}

L'arbitre a la possibilit\'e de faire sortir, temporairement ou 
d\'efini\-tivement, des joueurs du tournoi avec la commande `\verb|-|' 
du menu principal.  Le joueur est alors consid\'er\'e comme absent: il 
ne peut \^etre appari\'e pour cette ronde (s'il l'\'etait, son 
appariement est d\'etruit), il ne jouera pas et ne marquera pas de 
point contre Bip.  Il est possible de le faire revenir avec la 
commande~`\verb|+|'.  Dans la liste des joueurs (commande~`\verb|L|'), 
les joueurs ``absents'' sont indiqu\'es par le signe~`\verb|-|'.

	Bien que ces commandes soient principalement utilis\'ees quand
un joueur abandonne au milieu d'un tournoi, on peut leur imaginer
d'autres applications.  Par exemple, au moment de la finale, tous les
joueurs quittent le tournoi sauf deux.  Aussi PAPP offre-t-il une
commodit\'e pour faire sortir ou revenir tous les joueurs: apr\`es
`\verb|+|' ou `\verb|-|', tapez `\verb|*|' en r\'eponse au prompt. 

\subsection{Correction d'un r\'esultat}

Il arrive parfois que deux adversaires ne respectent pas les bonnes 
couleurs dans leur partie, ou qu'ils se trompent de scores en 
remplissant le coupon apport\'e \`a l'arbitre.  PAPP permet \`a 
celui-ci de corriger ces deux types d'erreurs.  Utilisez la commande 
`\verb|C|' du menu principal : PAPP vous demande de rentrer 
successivement le num\'ero de la ronde du coupon \`a corriger (ce peut 
\^etre le num\'ero de la ronde courante si vous avez commenc\'e \`a 
rentrer des r\'esultats pour cette ronde), le joueur noir 
\emph{correct} (c'est-\`a-dire dans la vraie partie), le \emph{vrai} 
score noir, le joueur blanc \emph{correct} puis le \emph{vrai} score 
blanc.  PAPP v\'erifie la coh\'erence de la correction apport\'ee 
(pr\'esence d'un appariement entre ces joueurs \`a cette ronde, 
l\'egalit\'e des scores, \emph{etc.}), affiche le coupon modifi\'e, 
\'ecrit le nouveau fichier interm\'ediaire sur le disque et recalcule 
les nouveaux scores et les nouveaux d\'epartages des joueurs dans le 
tournoi.

Note : PAPP ne refait pas les appariements de la ronde en cours suite 
\`a une correction dans une ronde pass\'ee ; c'est une d\'ecision qui 
d\'epend du r\`eglement du tournoi et d'autre param\`etres (les 
appariements ont-ils \'et\'e annonc\'es ?  les parties ont-elles 
commenc\'e ?)  et qui doit revenir \`a l'arbitre.  Si l'arbitre 
d\'ecide de recalculer de nouveaux appariements, il peut le faire en 
effacant tous les appariements (commandes `\verb|M|', puis `\verb|Z|' 
du menu principal) puis en demandant le recalcul des appariements 
automatiques (commandes `\verb|A|' du menu principal).

\subsection{Fourrager dans le fichier interm\'ediaire}

Ce n'est gu\`ere souhaitable, mais peut s'av\'erer indispensable pour 
les corrections impossibles \`a faire directement dans PAPP avec la 
commande `\verb|C|', par exemple pour faire une permutation circulaire 
des appariements si quatre joueurs se sont tromp\'es en m\^eme temps 
plusieurs rondes auparavant, ou pour effacer tout-\`a-fait un joueur 
inscrit par erreur.  Le fichier interm\'ediaire contient quelques 
indications pour guider les mortels qui oseraient s'y aventurer.  
Cherchez donc le bloc indiquant le r\'esultat de la ronde 
incrimin\'ee; les r\'esultats sont stock\'es sous la forme 
$$\hbox{\verb|(|{\it joueur\_noir score\_noir joueur\_blanc 
score\_blanc\/}\verb|);|}
 $$
Les ``r\'esultats'' contenant en eux-m\^emes les appariements, il 
n'est pas n\'ecessaire qu'ils soient coh\'erents avec le bloc des 
appariements situ\'e au-dessus (bloc qui peut d'ailleurs ne pas 
exister, surtout si vous avez choisi ``sauvegarde diff\'er\'ee''); il 
importe en revanche que le bloc des r\'esultats soit coh\'erent en 
lui-m\^eme; donc corrigez soigneusement la ligne erron\'ee, sinon PAPP 
signalera une erreur fatale et s'arr\^etera.  (Mieux encore: laissez 
la ligne originale en commentaire.  Si vous n'\^etes pas s\^ur de 
vous, faites une copie du fichier interm\'ediaire avant de le 
modifier).

\subsection{Utilisation du multit\^ache}

PAPP a \'et\'e d\'evelopp\'e initialement pour MiNT, le syst\`eme 
d'exploitation multit\^ache de l'Atari ST, avec l'objectif d'\^etre 
facilement portable sur d'autres plateformes.

Une premi\`ere cons\'equence de cela est que PAPP n'impose aucune 
restriction sur les noms de fichiers; vous n'\^etes limit\'e que par 
le syst\`eme de fichiers de votre machine (lequel peut distinguer 
majuscules et minuscules, c'est pourquoi PAPP ne convertira jamais un 
nom en majuscules, par exemple).

Si PAPP est compil\'e pour reconna{\^\i}tre les signaux, et si le 
shell conna{\^\i}t le ``job control'', alors il peut \^etre suspendu 
en tapant [Ctrl-Z] dans le menu principal.  Il r\'epond \'egalement 
\`a certains signaux: SIGINT est toujours ignor\'e; les signaux 
SIGHUP, SIGQUIT et SIGTERM terminent le programme ``proprement'' en 
sauvegardant son \'etat.  Les autres signaux tuent le programme sans 
que celui-ci puisse proc\'eder aux sauvegardes n\'ecessaires.

PAPP peut \'egalement s'ex\'ecuter dans une fen\^etre, sous MW ou 
TosWin (sur l'Atari) ou Xterm (sous X11); les dimensions de celle-ci 
sont obtenues par un ioctl(), et comme le signal SIGWINCH est 
g\'er\'e, on peut m\^eme redimensionner la fen\^etre pendant que le 
programme tourne!

PAPP peut \^etre lanc\'e avec un (unique) argument sur la ligne de 
commande; cet argument est alors interpr\'et\'e comme le nom du 
fichier de configuration \`a lire.  Par d\'efaut, ce fichier de 
configuration contient `\verb|#!  papp|' sur sa premi\`ere ligne, ce 
qui lui permet d'\^etre ``ex\'ecut\'e'' directement, sur certains 
syst\`emes d'exploitation, en particulier Unix et MiNT.

\subsection{Jeux autres qu'Othello}

Bien qu'initialement \'ecrit pour Othello $8\fois 8$, PAPP peut \^etre 
utilis\'e pour tout jeu o\`u il y a un nombre fixe de ``pions'' (ici, 
$64$) \`a se partager entre les deux joueurs.  Dans ce cas, il faut 
\'egalement pr\'eciser combien de pions fait Bip (c'est important pour 
le d\'epartage).  Par exemple, si on veut organiser un tournoi 
d'Othello $10\fois 10$ o\`u Bip fait $35$ pions sur cent, on \'ecrira 
simplement \verb|score-bip = 35 / 100|.

Si on omet `\verb|/100|', cela signifie qu'on veut modifier le score 
de Bip, mais pas le total de pions.  Ici en l'occurence, Bip 
marquerait $35$ pions sur $64$, donc gagnerait toutes ses parties!  
PAPP obtemp\'erera, mais vous avertira quand m\^eme que c'est un peu 
\'etrange\dots\ (Note: cette bizarrerie dispara{\^\i}tra peut-\^etre 
dans les versions ult\'erieures de PAPP.)

	Le ``total de pions'' situe sous la barre de fraction peut
m\^eme \^etre un nombre impair (songez au Go par exemple). Si le total
de pions est $\le2$, cela signifie qu'on a affaire \`a un jeu ``sans
pions'', tel que les echecs, et qu'il y a identit\'e entre un pion et un
point (ou un demi-point); dans ce cas, il n'y a qu'un seul d\'epartage
raisonnable, le {\em Buchholz\/} (somme des scores des adversaires, en
demi-points). 

	D'autre part, il sera sans doute n\'ecessaire de sp\'ecifier les
couleurs du premier et du second joueur, vu qu'\`a Othello ce sont les
Noirs qui commencent, \`a l'inverse de quasiment tous les autres jeux~;
sinon, les fiches individuelles n'indiqueront pas les bonnes couleurs.
Si vous inventez un jeu avec des pions rouges et verts et que ce sont
les Rouges qui commencent, pensez \`a ajouter la ligne
\begin{verbatim}
    Couleurs = { "Rouge", "Vert" };
\end{verbatim}
dans votre fichier de configuration.

	Enfin, dans certains jeux ``\`a pions'' comme le Go, on pr\'ef\`ere
comptabiliser la diff\'erence de pions entre les joueurs plut\^ot que le
nombre de pions du premier joueur. Il est toujours possible d'entrer les
r\'esultats sous forme relative ou absolue dans PAPP, mais il faut
pr\'eciser si vous voulez voir les r\'esultats affich\'es sous forme
relative ou absolue~; pour le Go, on ajoutera la directive
\begin{verbatim}
    Affichage-pions relatif;
\end{verbatim}
dans le fichier de configuration. Ceci affectera l'affichage des
r\'esultats et celui des fiches individuelles, mais le d\'epartage tel
qu'il appara{\^\i}t dans la liste des joueurs est toujours bas\'e sur le
score ``absolu'' des joueurs.

\section{Copyright}

	Ce programme (c'est-\`a-dire sources et documentation) est
propri\'et\'e de l'auteur, Thierry Bousch. Il peut \^etre librement
copi\'e et distribu\'e selon la GPL (GNU Public License)~; cette licence
est d\'etaill\'ee dans le fichier `\verb|COPYING|'.

	L'auteur d\'ecline toute responsabilit\'e concernant
l'utilisation de ce programme, en particulier les dommages directs ou
indirects qui pourraient \^etre caus\'es par celui-ci.  Le programme est
fourni tel quel, sans aucune garantie.

Je suis n\'eanmoins tr\`es int\'eress\'e par toute remarque, critique, 
compte rendu de bug, concernant PAPP. Vous pouvez me contacter par 
courrier \'electronique
(\verb|bousch@topo.math.u-psud.fr|) ou poster un message sur la liste 
discutant de l'avenir de PAPP (\verb|ffo-papp@egroups.fr|).

\newpage
\tableofcontents

\end{document}

