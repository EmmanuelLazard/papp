%
% Documentation de PAPP, en LaTeX.
%%
%\documentstyle[11pt]{article}
\documentclass[10pt]{article} 
\usepackage{a4wide}
%\usepackage{times}

%\french
\def\contentsname{Table of contents}
\parskip 1ex plus 0.5ex minus 0.5ex
\tolerance 1000
\frenchspacing

% Macro pour inserer du code TeX non utilise pour la version HTML
\def\tthdump#1{#1}

% Macro pour ecrire du code different en HTML (branche iftth)
% et en TeX (branche \else). Usage :
% \iftth  code HTML   \else   code Tex \fi
\newif\iftth

% \times semble manquer dans TtH, on le definit
\def\fois
{\iftth 
   {\begin{html}<font face=symbol> ¥ </font>\end{html}} 
\else 
   \times 
\fi}  

\voffset=-0.6in
\begin{document}

\title{\huge\bf Documentation for Papp 1.36}
\author{ {Thierry Bousch {\tt <bousch@topo.math.u-psud.fr>}}
\and {Emmanuel Lazard {\tt <Emmanuel.Lazard@katouche.fr>}}
}
\date{Septembre 2012}
\maketitle

\iftth   
{\begin{html}<small>Si les math&#233;matiques de ce document sont
        bizarres, changez l'encodage (MacRoman ou ISO-8859-1) dans 
        votre navigateur. Et si le probl&#232;me persite, vous devriez 
        trouver une solution ici : http ://hutchinson.belmont.ma.us/tth/manual/ </small>
 \end{html}} 
\else {}
\fi    


\section{Introduction}

	PAPP is a pairing program, that is, something which is supposed to 
simplify referees work during tournaments.  With it, they can 
inscribe players, save the results of the rounds, calculate pairings 
automatically and produce files used to calculate a international 
rating list.

	PAPP also offers a mini-database to allow players numbers 
management, and the pairing procedure is completely parametrable.

	PAPP is written in ANSI C, especially for the GCC. The manual has 
been written in \LaTeX. The lexical and syntactic analysis phases use 
Flex and Bison, from the ''Free Software Foundation''.

\section{First contact}

Executable is called \verb|papp|, \verb|papp.exe| or \verb|Papp MacOS X| depending on the OS. When launched, PAPP looks for a configuration file and prints a message. It then looks for an ongoing tournament (presence of a papp work file, see section~\ref{inter}). If it is the case, PAPP
displays its caracteristics otherwise it asks for them. Then the main menu appears.

	All commands are accessible with a single key stroke, there is no upper/lower case distinction. 
It is not  necessary to type [Return] afterwards, nor is it necessary after the 
other menus, nor after the Yes/No questions.

	Experiment a little bit, it is not dangerous.  PAPP will emit a 
protestation beep if you press an invalid key (raise the sound level 
of your computer if necessary). Most commands will be uneffective 
since no player is inscribed in the tournament, so type `\verb|I|' to 
inscribe players.

\subsection{Inscriptions}

Faces with the prompt \verb|Nom & Prenom :|, you can start to type in the name of player for example `\verb|willi|', then press the [TAB] key ; PAPP then displays the list of all players whose name starts with  `willi', thats is \verb|WILLIAM (Jacquart)|, \verb|WILLIAMS Anthony|, \verb|WILLIAMS Eddie| and \verb|WILLIAMSON Tim|.  You'll notice that the entry buffer now holds \verb|WILLIAM|, because PAPP completed the name until the first ambiguity --- here, it does not know wether the fourth letter is `S' 
or a space.  Delete the buffer with [Backspace] ou [Esc], and type 
\verb|laz|, then [Tab].  This time, there is only one possible 
completion: \verb|LAZARD Emmanuel|, and you only have to confirm with 
[Return].

	Now, enter the name of a new, unknown player (use [Tab]
to verify that he/she is not already in the database).  PAPP will
ask for confirmation (you may have done a typographic error), then 
the player's nationnality (just hit [Return] if it is the default 
country) so that he can choose a number for the new player  --- you 
can change that number if you wish.  The player will then be added to 
the database as a new player.

Spend some time playing with the database and inscribing players.  
Once you get bored, just type [Return] (or Ctrl-C) at the prompt.  
Then, at the main menu, you can check the list of players in the 
tournament with `\verb|L|'.


\subsection{Pairings}

If you hit `\verb|V|', you'll see the list of all these players, one 
per line.  That's because they are not yet paired.  It is possible to 
pair them manually with `\verb|M|', or to ask PAPP to pair 
automatically all the players that are not yet paired with `\verb|A|'.  
So type `\verb|A|': the list of pairings appears shortly alfterwards; 
This time, there are two players on each line (except on the last line 
if there is an odd number of players).  This means the pairings are 
complete.

If some players show up late after the beginning of round one, the 
tournament director has the choice to accept them or not; but as far 
as PAPP is concerned, there is no difficulty: you come back to the 
main menu, you inscribe the players, and you type `\verb|A|' to 
complete the pairings.  It is possible to inscribe players for a 
particular round as long as all the results for that round have not 
been entered and validated, which should be flexible enough for most 
cases.

\subsection{Entering results}

When the games of the round end and you begin to collect score coupons 
on your desk, select the option `\verb|R|' to enter the results.  the 
prompt \verb|Joueur & Score:| is displayed in an empty screen; to 
enter a game result, type in the name (or the beginning of the name , 
or the number) of one of the players, followed by his/her score, 
separated by a space.  PAPP verifies the name or the number of the 
designated player, outputs a warning message if the result had already 
been entered, then displays the filled coupon. In case of an error, you can {\em remove} a coupon by typing the sign `\verb|-|' followed by the name of one of the players ( if you prefer by typing a player ID followed by the letter `\verb|e|' --- spacing between them is optional. It is possible to enter relative scores: for example, `\verb|145+4|', `\verb|145=|' and 
`\verb|145-6|' means that player 145 won by 4 discs, drew, or lost by 6 discs respectively. --- Here again, spacing is unnecessary as there is no ambiguity.

If you are not sure of the information on the coupon (name spelling or ID of a player), or if the result was orally given to you, press [TAB]: PAPP will display all coupon, first those already entered and then those left to enter.
Therefore, even if the list is too long to fit on a screen, at least the end of the list will still show. Another less convenient solution is to come back to the main menu and to choose the `\verb|V|' option to see all pairings and partial results.

PAPP will warn you when all results are entered. Proofread yourself (eventually by using [TAB] to check old results) and if everything seems correct, simply press `\verb|!|' in response to the prompt then  [Return]; after saving the results, PAPP will display the new standings.

For testing purposes, a `\verb|?|' option has been added instead of entering a result. Then PAPP will randomly generate results for still pending ones. If PAPP knows the players ratings, random results will follow winning probabilities.




\subsection{Chargement et sauvegarde} \label{inter}

	Il n'y a pas d'option ``Charger'' ou ``Sauver''.  Les
sauvegardes sont faites quand c'est nécessaire (voir le chapitre
``Mode de sauvegarde'' pour plus de détails) et lorsqu'on quitte le
programme.  Quand on relance PAPP, celui-ci relit le fichier
intermédiaire précédemment créé sur le disque (celui-ci
aura pour nom \verb|papp-internal-workfile.txt|, typiquement) et se remet exactement
dans l'état o\`u il était la dernière fois qu'on l'a quitté.

	Ceci a une conséquence importante~: avant un nouveau tournoi,
il faut penser à effacer ce fichier du disque, sinon PAPP se remettrait
dans l'état ou il était après la dernière ronde du tournoi
précédent, ce qui n'est généralement pas l'effet recherché.

\subsection{Appariements manuels}

Il est possible de forcer manuellement les appariements avec 
`\verb|M|'.  Pour des raisons de place disponible sur l'écran, il ne 
m'a pas semblé possible de donner les noms complets des joueurs ``en 
clair''.  Les joueurs sont donc représentés par les huit premiers 
caractères de leur nom, suivi de leur numéro (entre 
parenthèses)~; heureusement, il est possible d'afficher la liste ``en 
clair'' des joueurs et des appariements sans revenir au menu 
principal, puisque les options `\verb|L|', `\verb|V|' et `\verb|F|' 
sont également disponibles dans ce sous-menu.

	PAPP affiche d'abord les couples de joueurs appariés (comme
toujours, le premier joueur a les noirs), puis les joueurs non
appariés.  Ces deux listes sont mises à jour après chaque
opération.  Il y a trois opérations possibles~:

\begin{itemize}

	\item[`{\tt Z}' :] détruit tous les appariements. Cette commande ne
prend aucun argument, mais PAPP vous demandera confirmation.

\item[`{\tt D}' :] découpler un joueur. Cette commande prend un 
argument, qui est le nom ou le numéro du joueur.  Si ce joueur 
était apparié, son appariement est détruit.  Sinon, la commande 
est sans effet.  Comme toujours lorsque PAPP vous demande un nom de 
joueur, vous pouvez entrer seulement les premières lettres de son 
nom, ou obtenir une complétion du tampon d'entrée en appuyant sur 
la touche [Tab].

\item[`{\tt A}' :] apparier deux joueurs.  PAPP vous demande 
successivement le nom (ou le numéro) du joueur noir, puis du joueur 
blanc.  Si ces joueurs étaient déjà appariés, leurs 
précédents appariements sont détruits.

\end{itemize}

\subsection{Lecture des résultats}

	La commande `\verb|L|' donne le score et le départage des
joueurs à un moment donné du tournoi~; les données affichées ne
seront modifiées qu'après que les résultats de la ronde auront
été entièrement entrés et validés.  Le format d'affichage est
le suivant~: le rang dans le tournoi, le score puis le départage entre
crochets, un éventuel `\verb|-|' pour indiquer que le joueur est
absent (i.e., a abandonné le tournoi), puis le nom, le numéro et la
nationalité du joueur.

Il est possible d'obtenir les résultats individuels d'un joueur avec 
la commande `\verb|F|' du menu principal.  PAPP affiche alors les 
appariements et les résultats des rondes précédentes, sous la 
forme suivante~: le numéro de ronde, la couleur du joueur, puis le 
score de sa partie (eventuellement précédé d'un `\verb|+|' pour 
indiquer une victoire ou un `\verb|=|' en cas de nulle), et enfin le 
nom et le numéro Elo de son adversaire.  Si le fichier des joueurs 
contient des renseignements supplémentaires sur le joueur, ils sont 
affichés tels quels en bas de la fiche.  Pour PAPP, ce ne sont que 
des commentaires~: aucune interprétation n'est faite.  Ces 
commentaires sont également affichés lors de l'inscription du 
joueur.

Il est également possible d'afficher des résultats par équipe 
avec la commande `\verb|T|' du menu principal.  PAPP considère 
simplement que tous les joueurs d'un même pays font partie de la 
même équipe, et affiche donc un classement par équipe dans 
lequel le score de chaque équipe est obtenu en sommant les scores 
individuels des joueurs du tournoi ayant cette nationalité.

	Si vous avez entré seulement une partie des résultats d'une
ronde, vous pouvez vérifier ce que vous avez déjà entré avec la
commande `\verb|V|' du menu principal (encore une fois, le premier nommé
a les noirs).

\subsection{Autres commandes}


La commande `\verb|E|' du menu principal produit un fichier Elo qui 
peut être relu par JECH. A n'utiliser qu'à la fin du tournoi. Vous aurez la possibilité
d'ajouter des résultats à ce fichier, comme les scores de finales ou matchs de départage.

La commande `\verb|C|' permet de corriger un résultat erroné dans 
une ronde passée.  Pour plus de détails, voyez le chapitre 
``Correction d'un résultat'' plus loin dans ce manuel.

La commande `\verb|S|' permet de sauvegarder le classement dans un 
fichier.  Si le nom de fichier commence par un `\verb.|.'  (``pipe''), 
celui-ci est interprété comme une redirection vers une commande 
(cette facilité n'est pas disponible sur tous les systèmes 
d'exploitation).  De même, les commandes [Ctrl-V] et [Ctrl-F] 
permettent de rediriger vers un fichier (ou un pipe) la liste des 
appariements et la fiche d'un joueur, respectivement.  Si vous 
désirez une sauvegarde automatique, ronde par ronde, des résultats 
des parties et des classements intermédiaires, voyez le chapitre 
``Récapitulatif ronde par ronde'' plus loin dans ce manuel.

	La commande `\verb|X|' quitte le programme, en faisant toutes les
sauvegardes nécessaires.

	La commande [Ctrl-Z] suspend le programme, si le système
d'exploitation connaît le ``job control''.

	La commande `\verb|&|' affiche un petit message de copyright, le
numéro de version, la date de compilation, et diverses limitations
statiques du programme.

	Les commandes `\verb|-|' et `\verb|+|' sont utilisées quand un
joueur quitte le tournoi, ou revient dans le tournoi après l'avoir
quitté.  Il est possible de faire sortir ou revenir tous les joueurs
en tapant `\verb|*|' en réponse au prompt.  

    La commande `\verb|W|' propose d'abord de sauvegarder un tableau
complet de tous les résultats au format HTML. Pour chaque joueur,
on génère une ligne avec son classement, son nom, son pays,
une liste de cellules contenant le détail de chaque ronde
(couleur, adversaire, score, points marqués et cumul de points),
son total de points et son départage (pour le détail de chaque
cellule, voir plus loin le chapitre correspondant). On propose ensuite de
générer un tableau complet au format texte (et éventuellement de
le sauvegarder) après n'importe quelle ronde. On affiche le numéro
du joueur; pour chaque ronde la couleur, le résultat (gain/nulle/perte),
l'adversaire; le total de points et le départage. Dans le fichier
texte, on sauvegarde en plus le total de pions, le Buchholz, la suite
de couleurs jouées (et le total, négatif si on a joué plus de fois
les noirs, positif si on a joué plut\^ot les blancs) et le flottement
éventuel à la dernière ronde ({\em u} pour {\em up}, on a flotté
vers la haut~; {\em d} pour {\em down}, on a flotté vers le bas).
Dans tous les cas, le départage affiché est le {\em Brightwell}
(voir le chapitre correspondant)~; le total de pions et le Buchholtz
sont les valeurs exactes, sans tenir compte des règles spéciales
du départage de Brightwell (par exemple, si on joue Bip, le Buchholtz
ne change pas et le total de pions est augmenté du score contre Bip
indiqué dans le fichier de configuration). Si un joueur a abandonné
avant le fin du tournoi, le départage calculé dans le tableau complet
au format texte tiendra compte de cela, quelle que soit la ronde affichée.

Enfin, la commande `\verb|N|' permet de modifier la nationalité d'un
joueur. Cela peut par exemple être utile si l'on veut créer des
équipes non strictement nationales~: imaginons que nous ayons six 
joueurs américains que l'on veut répartir en deux équipes de trois, 
il suffit de changer pour la durée du tournoi la nationalité de 
trois d'entre eux en `\verb|USA2|' pour que le classement par équipe
fasse la distinction entre les équipes `\verb|USA|' et `\verb|USA2|'.





\section{Automatic pairings}

That is what behind the `\verb|A|' command from the main menu. First, we insist that `\verb|A|' just completes
already existing pairings; in no way can this command modify already chosen pairings, by PAPP or the user. If pairings are already completed, `\verb|A|' has no other effect to display the pairings list.

Pairings are always calculated by optimisation, even for round 1. First if there is an odd number of players, a fictional player, BYE, is added. So it can be assumed there is an even number of players. Then for each couple of players $(i,j)$, a {\em penalty} is computed --- that is a real positive number $p_{ij}$ expressing how much it is not desirable that player $i$ plays black against player $j$.
PAPP then looks for the pairing  $(i_1,j_1)$, $(i_2,j_2)$, \dots, $(i_n,j_n)$, minimizing overall penalty
 $$ {\it Overall\_Penalty} = \sum_{k=1}^n p_{i_k j_k} $$

This a classical graph theory problem (maximum-weight perfect matching) that can be solved in polynomial time, typically $O(n^3)$. Papp implementation seems to rather be in $O(n^4)$.

\subsection{Penalties decription}

Penalties are computed as follows:

 $$\matrix{
    p_{ij} &= {\it pen\_color}(i,{\rm BLACK})
	    + {\it pen\_color}(j,{\rm WHITE})		\hfill\cr
	   &\quad{}+ {\it pen\_float}(i,j)		\hfill\cr
	   &\quad{}+ {\it pen\_repetition}(i,j)		\hfill\cr
	   &\quad{}+ {\it pen\_chauvinism}(i,j)	\hfill\cr
	   &\quad{}+ {\it pen\_elitism}(i,j)	    \hfill\cr
 }$$


\subsubsection{Color penalties}

	The term ${\it pen\_color}(i,{\it color})$ indicates that it is not desirable for player $i$ to play with color $\it color$ this round. Let $\delta_i$ be the {\em chromatic difference\/} of the player, that is number of times he played Black minus number of times he played White.
	Depending on whether he plays Black or White at this round, its chromatic difference will become $\delta_i+1$ or $\delta_i-1$, respectively. We then use:

 $$ {\it pen\_color}(i,{\it color}) =
    {\bf p\_coul}\Bigl[\left|\delta_i\pm1\right|\Bigr]
  + {\it pen\_color\_repeated} $$
Term $\it pen\_color\_repeated$ is $\bf p\_repcl$ if the player has already played that color at the previous round and  $0$ otherwise. ${\bf p\_coul}[0]$ is always $0$.

\subsubsection{Floating penalties}

The term ${\it pen\_float}(i,j)$ indicates that it is not desirable to have a pairing between two players whose scores are too different. We here use:
 $${\it pen\_float}(i,j) =
    {\bf p\_flot}[f] + {\it corr\_flot}(i) + {\it corr\_flot}(j)
 $$
where $f=\left| S_i-S_j \right|$ is the score difference between the players, expressed in half-points.
We should always have ${\bf p\_flot}[0]=0$.

Corrective term ${\it corr\_flot}(i)$ is $+{\bf p\_flcum}$ if the player floats in the same direction as the previous round. It values $-{\bf min\_fac}$ if the player floats in the opposite direction. It is always $0$ if the player does not float this round or has not floated at the previous one, or for the first round.

A player paired against BYE always floats down (see below to see how much). Corrections, if negative cannot exceeded (in absolute value) the term they are correcting, so that ${\it pen\_float}(i,j)$ will always be positive.

\subsubsection{Repetition penalties} 

The term ${\it pen\_repetition}(i,j)$ prevents two players to be paired again if they have already done so i a previous round. If $i$ and $j$ have already played together with the same colors, the $\bf p\_mcoul$ penalty is added; if they have already played together with opposite colors, $\bf p\_clopp$ is added.
If they have already played together several times, above penalties get added.
The $\bf p\_desuite$ penalty is added if both players have met at the previous round.

\subsubsection{Chauvinisme penalty}
The term ${\it pen\_chauvinisme}(i,j)$ penalizes games between players from the same country. Its value is ${\bf p\_chauv}[{\it
round\_number}]$ if $i$ and $j$ have the same nationality, and $0$ otherwise.


\subsubsection{Elitisme penalty} 
The term ${\it pen\_elitisme}(i,j)$ says that, when games between players having different scores
are necessary, it is better to have these matches at the lower end of the standings instead of the upper end.
We use:
$${\it pen\_elitisme}(i,j) = 
({\bf p\_elit}[{\it round\_number}] \fois (S_i+S_j) \fois f ) / 2
 $$
where $f=\left| S_i-S_j \right|$ is the score difference between the players, expressed as half-points.


\subsubsection{BYE case} 

If one of the player is BYE, penalties simplify because there is no worry about colors.
We use:

 $$\matrix{
   p_{i,{\rm BYE}} = p_{{\rm BYE},i} &=
   	{\it pen\_color}(i, {\rm GREY})		\hfill\cr
	&\quad{}+{\it pen\_float}(i,{\rm BYE})	\hfill\cr
	&\quad{}+{\it pen\_repetition}(i,{\rm BYE})	\hfill\cr
	&\quad{}+{\it pen\_elitisme}(i,{\rm BYE})	\hfill\cr
 }
 $$
 
 \begin{itemize}
 
 \item
As the chromatic difference is not modified when playing BYE, color penalty is:
$$
  {\it pen\_color}(i,{\rm GREY}) =
    {\bf p\_coul}\Bigl[\left| \delta \right|\Bigr]
$$
 
 \item
	$\it pen\_float$ and $\it pen\_elitisme$ penalties are defined as above;
	but even as BYE score is zero, 
	sont définies comme plus haut~; cependant, bien que le score de 
	Bip soit de zéro point, we act ``as if" his score was immediately lower, by half a point, than that of the last
player present (to prevent excessive floating penalty). So if $f = S_i - 
	S_{\rm min} + 1$~:
	
$$ 
  {\it pen\_float}(i,{\rm BYE}) =
    {\bf p\_flot}[f] 
$$
$$
  {\it pen\_elitisme}(i,{\rm BYE}) =
    ({\bf p\_elit}[{\it round\_number}] \fois (S_i+S_{\rm min} - 1) 
     \fois f) / 2 
$$ 
 \item
	In the $\it pen\_repetition$ penalty, distinction between $\bf
p\_mcoul$ et $\bf p\_clopp$ disappears, there is just one penalty to consider: $\bf p\_bipbip$ is added as many times as the player has played against BYE in the previous rounds, and as above, $\bf p\_desuite$ is added if the player has just plkayed BYE at the previous round.


\end{itemize}

\subsection{Conditions concerning penalties}

For coherence, PAPP imposes a minimum number of conditions on penalties. If they are not fulfilled, the program abord on a fatal error. The conditions are the following:

\begin{itemize}

\item	Color penalties must grow along chromatic difference:
$$ 0 = {\bf p\_coul}[0] \le {\bf p\_coul}[1] \le {\bf p\_coul}[2] \le\cdots$$

\item	Floating penalties must grow with score difference and:

\item	${\bf min\_fac}$ must be lower or equal to half the smallest floating penalty:
$$ 0 = {\bf p\_flot}[0] \le 2\fois{\bf min\_fac} \le {\bf p\_flot}[1] \le
	{\bf p\_flot}[2] \le \cdots $$

\end{itemize}

These conditions do not assure ``reasonable" pairings. Floating and color penalties should also grow ``fast enough". The last condition ensures that the floating penalty after correction is always positive.

\subsection{Last pass}

Once optimisation is done, PAPP checks if there exists other optimal pairings that can be deduced from the first one 

Once the optimization has been carried out, PAPP checks whether there are other optimal pairings,
deduced from the first only by color inversions. When this is the case ---
that is to say for any pair $\{i,j\}$ of matched players such
that $p_{ij}=p_{ji}$ ---, PAPP will choose the colors as follows: we
search for the most recent round where $i$ and $j$ had different colors
 and we invert these colors. If it is not possible,
that is to say if both players have always played with the
same colors, then we draw the colors randomly.

We insist on the fact that this last pass does not degrade pairing quality; PAPP will only chose etween several optimal pairings.


\section{Le fichier de configuration}

% Macros pour la grammaire
% Laisser les espaces : ils sont transparents en TeX, 
% mais utiles en HTML.
%
\def\gram#1{$\langle\hbox{\it#1}\rangle$}  
\def\opt#1{$[\hbox{\it#1}]$}     % TtH ne connait pas \lbrack et \rbrack !
\def\optr#1{$\lbrace\hbox{\it#1}\rbrace$}
\def\fleche{$  \longrightarrow\,\,\,  $}
\def\ou{$\vert\,\tthdump{\,}$}
\def\vrb#1{{\tt #1}}	% le verbatim du pauvre!

	Ce fichier est chargé par PAPP dès l'exécution.  Le nom
de ce fichier est déterminé à partir de la ligne de commande, si
celle-ci contient exactement un argument.  Sinon, PAPP utilise la
variable d'environnement \verb|PAPP_CFG|, si celle-ci est définie~;
sinon, PAPP prendra \verb|papp.cfg| comme nom de fichier (dans le
répertoire courant).

	Si PAPP ne peut ouvrir le fichier de configuration, il
créera un fichier \verb|papp.cfg| dans le répertoire courant,
contenant ses réglages par défaut mais prévient l'utilisateur
en affichant un message.
 
	Ce fichier de configuration contient les noms des autres
fichiers requis par PAPP, la liste des pénalités, et quelques
autres paramétrages.
 
\subsection{Conventions lexicales}

	Le texte est découpé en chaînes, entiers, mots-clés
et caractères isolés.  Les espaces, tabulations, retours chariot
et passages à la ligne sont ignorés~; c'est le point-virgule qui
tient lieu de fin de commande (voir plus bas).

	Une \gram{chaine} est une chaîne de caractères
délimitée par des doubles guillemets (\verb|"|), tenant sur une
seule ligne.

	Un mot-clé est une séquence réservée de lettres et de
tirets (`\verb|-|'), le tiret ne pouvant pas apparaître comme
premier caractère.  Aucune distinction n'est faite entre majuscules
et minuscules.

	Un \gram{entier} désigne un nombre entre $0$ et $2^{31}-1 =
2\,147\,483\,647$~; ce dernier nombre peut également être obtenu
avec le mot-clé \verb|INFINI| --- comme tous les mots-clés, on
pourrait aussi l'écrire en minuscules, mais on n'aurait plus
l'impression qu'il s'agit d'un grand nombre.  (Note~: cette limite
dépend en réalité de l'implémentation~; un \gram{entier} étant
représenté en C par un \verb|signed long|, la limite serait
$2^{63}-1$ sur les machines 64-bits telles que le DEC Alpha.)

	Si un `\verb|%|' ou `\verb|#|' apparaît sur une ligne, tout
le reste de la ligne est ignoré (ainsi que le caractère lui-même),
sauf si le `\verb|%|' est immédiatement suivi par un `\verb|_|'~; la
séquence `\verb|%_|' est ignorée (elle équivaut à un espace),
donc le reste de la ligne n'est {\em pas\/} ignoré.

	Si le mot \verb|__eof__| apparaît dans le fichier, tout
ce qui suit est ignoré (le fichier est immédiatement refermé).

\subsection{Grammaire}

La syntaxe du fichier de configuration est la suivante~:

\medbreak

\halign{\noindent\hskip 1cm\relax#\hfil\cr
    \gram{fichier-config} \fleche \optr{\gram{commande}  \vrb;}			\cr
    \gram{commande} \fleche \gram{vide}						\cr
	\qquad \ou \vrb{fichier} (\vrb{joueurs} \ou \vrb{nouveaux} \ou \vrb{inter}) 
	\vrb{=} \gram{chaine}	\cr
	\qquad \ou \vrb{fichier} (\vrb{appariements} \ou \vrb{resultats} \ou \vrb{classement}) 
	\vrb{=} \gram{chaine}	\cr
	\qquad \ou \vrb{fichier} (\vrb{equipes} \ou \vrb{tableau-croise}) 
	\vrb{=} \gram{chaine}	\cr
	\qquad \ou \vrb{pays =} \gram{chaine}				\cr
	\qquad \ou \vrb{score-bip =} \gram{entier} \opt{\vrb/ \gram{entier}}	\cr
	\qquad \ou \vrb{sauvegarde} (\vrb{immediate} \ou \vrb{differee})	\cr
	\qquad \ou \vrb{XML} (\vrb{true} \ou \vrb{false})	\cr
	\qquad \ou \vrb{dossier} (\vrb{true} \ou \vrb{false})	\cr
	\qquad \ou \vrb{impression} (\vrb{manuelle} \ou \vrb{automatique} \gram{entier})	\cr
	\qquad \ou \vrb{affichage-pions} (\vrb{absolu} \ou \vrb{relatif})	\cr
	\qquad \ou \vrb{couleurs = \char`\{} \gram{chaine} \vrb, 
	 \gram{chaine} \vrb{\char`\}}	\cr
	\qquad \ou \vrb{zone-insertion} \opt{chaine}
		\vrb= \gram{entier} \vrb- \gram{entier}				\cr
	\qquad \ou \gram{indication-penalites} \ou \gram{commande-interne}   	\cr
%    \gram{type-fichier} \fleche \vrb{joueurs} \ou \vrb{nouveaux} \ou \vrb{inter}	\cr
%    \gram{type-sauvegarde} \fleche \vrb{immediate} \ou \vrb{differee}		\cr
}

\medbreak

Les deux commandes suivantes sont acceptées pour des raisons de compatibilité avec d'anciennes versions (antérieures à la 1.30) des fichiers de configuration.

\medbreak

\halign{\noindent\hskip 1cm\relax#\hfil\cr
	\qquad \ou \vrb{brightwell =} \gram{entier}				\cr
	\qquad \ou \vrb{toutes-rondes} \vrb= \gram{entier}
		\vrb- \gram{entier} \vrb{joueurs}				\cr
}

\medbreak

\subsection{Liste des pénalités}

La liste des pénalités se décompose en cinq sections~: les 
pénalités de couleur, celles de flottement, celles de 
répétition, celles de chauvinisme et celles d'élitisme.  Chaque 
section commence par un label indiquant de quelle section il s'agit, 
puis la liste des pénalités de cette section.

\medbreak
\halign{\noindent\hskip 1cm\relax#\hfil\cr
    \gram{indication-penalites} \fleche \vrb{penalites \char`\{} \optr{section}
     \vrb{\char`\}}	\cr
    \gram{section} \fleche \gram{section-couleur} \ou \gram{section-flottement} \cr
	\qquad \qquad \qquad  \ou \gram{section-repetition} \ou \gram{section-chauvinisme} \cr
	\qquad \qquad \qquad  \ou \gram{section-elitisme} \cr
}
\medbreak

\subsubsection{Pénalités de couleur}

La section ``couleur'' permet de choisir les penalités ${\bf p\_coul}
[\delta]$ et $\bf p\_repcl$. La syntaxe est

\medbreak
\halign{\noindent\hskip 1cm\relax#\hfil\cr
    \gram{section-couleur} \fleche \vrb{Couleur :}
		\optr{\gram{penalite-couleur} \vrb;}	\cr
    \gram{penalite-couleur} \fleche \gram{entier} \opt{\vrb+}
		\vrb{fois =} \gram{pen}			\cr
	\qquad \qquad \qquad \qquad \qquad \ou \vrb{de-suite =} \gram{pen}		\cr
    \gram{pen} \fleche un \gram{entier} entre $0$ et $10\,000\,000$	\cr
}
\medbreak

\noindent Considérons l'exemple suivant~:
\begin{verbatim}
    Couleur :
        2  fois  =  500;  % delta == 2
        3+ fois  = 5000;  % delta >= 3
        de-suite =  100;  % p_repcl
\end{verbatim}
Cela signifie qu'on aura une penalité de $500$ points si l'écart de
couleur ($\delta$) vaut deux, et $5000$ s'il vaut trois {\em ou plus\/}~; et
il y a $100$ points de pénalité si l'on joue deux fois de suite avec la
même couleur.

\subsubsection{Pénalités de flottement}

La section ``flottement'' concerne les variables ${\bf p\_flot}(f)$,
$\bf p\_flcum$ et $\bf min\_fac$. La syntaxe est

\medbreak
\halign{\noindent\hskip 1cm\relax#\hfil\cr
    \gram{section-flottement} \fleche \vrb{Flottement :}
		\optr{\gram{penalite-flottement} \vrb;}	\cr
    \gram{penalite-flottement} \fleche \gram{entier} \opt{\vrb+}
		\vrb{demi-point =} \gram{pen}		\cr
	\qquad \qquad \qquad \qquad \qquad \ou \vrb{de-suite =} \gram{pen}		\cr
	\qquad \qquad \qquad \qquad \qquad \ou \vrb{minoration =} \gram{pen}		\cr
}
\medbreak

\noindent On notera que \verb|demi-point| peut prendre un `s' au pluriel.
Voici un exemple~:
\begin{verbatim}
    Flottement :
        1  demi-point  =  100;  % p_flot[1]
        2  demi-points =  500;  % p_flot[2]
        3+ demi-points = 5000;  % p_flot[3 et plus]
        de-suite       =   50;  % p_flcum
        minoration     =   10;  % min_fac
\end{verbatim}

\subsubsection{Pénalités de répétition}

Ensuite vient la section ``répétition'', qui concerne les variables
$\bf p\_mcoul$, $\bf p\_clopp$, $\bf p\_bipbip$ et $\bf p\_desuite$. Ces
variables sont initialisées par les déclara\-tions suivantes,
respectivement~:

\medbreak
\halign{\noindent\hskip 1cm\relax#\hfil\cr
    \gram{section-repetition} \fleche \vrb{Repetition :}
    		\optr{\gram{penalite-repetition} \vrb;}		\cr
    \gram{penalite-repetition} \fleche \vrb{memes-couleurs =} \gram{pen}	\cr
	\qquad \qquad \qquad \qquad \qquad \ou \vrb{couleurs-opposees =} \gram{pen}		\cr
	\qquad \qquad \qquad \qquad \qquad \ou \vrb{bip-bip =} \gram{pen}			\cr
	\qquad \qquad \qquad \qquad \qquad \ou \vrb{de-suite =} \gram{pen}			\cr
}
\medbreak

\subsubsection{Pénalités de chauvinisme}

Le tableau ${\bf p\_chauv}[r]$ est initialisé par la section
``chauvinisme'', de la manière suivante~:

\medbreak
\halign{\noindent\hskip 1cm\relax#\hfil\cr
    \gram{section-chauvinisme} \fleche \vrb{Chauvinisme :}
    		\optr{\gram{penalite-chauvinisme} \vrb;}		\cr
    \gram{penalite-chauvinisme} \fleche \vrb{ronde} \gram{entier} \opt{\vrb+}
    	\vrb= \gram{pen}	\cr
}
\medbreak

Les rondes sont numérotées à partir de un. Si par exemple on veut
que la pénalité de chauvinisme soit égale à $100$ pour les dix
permières rondes et à $1000$ ensuite, on écrira~:
\begin{verbatim}
    Chauvinisme :
        ronde  1+ =  100;  % ronde 1  et suivantes
        ronde 11+ = 1000;  % ronde 11 et suivantes
\end{verbatim}

\subsubsection{Pénalités d'élitisme}

Le tableau ${\bf p\_elit}[r]$ est initialisé par la section
``élitisme'', de la manière suivante~:

\medbreak
\halign{\noindent\hskip 1cm\relax#\hfil\cr
    \gram{section-elitisme} \fleche \vrb{Elitisme :}
    		\optr{\gram{penalite-elitisme} \vrb;}		\cr
    \gram{penalite-elitisme} \fleche \vrb{ronde} \gram{entier} \opt{\vrb+}
    	\vrb= \gram{pen}	\cr
}
\medbreak


Les rondes sont numérotées à partir de un.  Si par exemple on 
veut que la pénalité d'élitisme soit égale à $5$ pour les 
cinq permières rondes, à $25$ pour les rondes six à dix, et à 
$100$ ensuite, on écrira~:
\begin{verbatim}
    Elitisme :
        ronde  1+ =   5;  % ronde 1  et suivantes
        ronde  6+ =  25;  % ronde 6  et suivantes
        ronde 11+ = 100;  % ronde 11 et suivantes
\end{verbatim}

\subsection{Remarques diverses}

	Dès que PAPP rencontre le mot-clé \verb|penalites|, il remet
à zéro {\em toutes\/} les pénalités. Ceci permet de partir sur
des bases saines, en évitant que les pénalités définies par
l'utilisateur ne viennent se mélanger avec celles par défaut. En
particulier, si l'utilisateur omet de déclarer une pénalité,
celle-ci restera égale à zéro. Ce système garantit également une
certaine compatibilité ascendante entre les différentes versions de
PAPP.

	Pour des raisons historiques, les pénalités ne peuvent
excéder 10 millions~; les pénalités plus grandes seront ramenées
à cette valeur (avec un message d'avertissement).

\subsection{Tournois toutes-rondes}
 
	Bien qu'initialement prévu pour organiser des tournois selon
le système suis\-se (ou l'une de ses multiples variantes), PAPP peut
également organiser des tournois toutes-rondes ou plusieurs fois
toutes-rondes. Lors du lancement d'un nouveau tournoi, PAPP vous demandera le nombre de rondes.
Il saura alors qu'il doit effectuer un toutes-rondes s'il n'y a pas trop de joueurs inscrits.
Si ce n'est pas possible, c'est-à-dire s'il y a trop
de joueurs, ou bien si des joueurs ont quitté le tournoi ou sont
entrés après la fin de la première ronde, ou encore si vous avez
forcé certains appariements à la main, alors PAPP se repliera sur
l'algorithme d'optimisation décrit plus haut. 

\subsection{Le départage de Brightwell}

	Au démarrage d'un nouveau tournoi, PAPP vous demande la valeur du coefficient de Brightwell pour le calcul du départage. Vous pouvez spécifier $0$ pour avoir un départage au nombre de pions. Si au contraire vous spécifiez une valeur strictement positive (éventuellement décimale),
celle-ci sera prise comme valeur du coefficient $\beta$ de Brightwell. 
Rappelons que le départage de Brightwell est donné par
 $${\it departage} = {\it nombre\_pions} + \beta\fois{\it buchholz}$$

Dans cette formule, le $\it buchholz$ est la somme des scores des adversaires rencontrés (en comptant un point par victoire). Une partie non-jouée (parce
que le joueur est arrivé en retard ou a abandonné), une partie
contre Bip, ou une partie contre un adversaire qui a abandonné
le tournoi, est remplacée par une partie nulle contre soi-même,
avant d'appliquer la formule ci-dessus.

Imaginons par exemple un joueur $A$, qui joue et gagne 33/31 contre 
$B$ à la première ronde, et se retrouve contre Bip à la 
deuxième ronde.  On suppose que $B$ a gagné sa deuxième partie.  
Si on prend comme coefficient de Brightwell $\beta=6$, quel est le 
départage de $A$ à l'issue de la deuxième ronde~?

On constate que $A$ a 2 points et $B$, 1 point.  La partie de $A$ contre Bip doit être 
remplacée par une partie nulle 32/32 contre $A$.  Le départage 
vaut donc
 $${\it departage} = (33+32) + 6 \fois (1+2) = \hbox{83 pions}.$$

%Insistons sur le fait que le départage de Brightwell est calculé 
%dans PAPP à partir de la somme {\it en demi-points} des score des 
%adversaires rencontrés, tandis que les règlements écrits des 
%tournois fixent généralement la valeur du coefficient de 
%Brightwell pour des scores en points entiers.  Dans ce cas, la valeur 
%du coefficient de Brightwell spécifiée dans le fichier de 
%configuration doit être {\it la moitié} de celle apparaissant dans 
%le règlement.

%Notes~: les commandes `\verb|-|' et `\verb|+|' ont un effet de bord 
%sur le départage de Brightwell, parce que les joueurs absents (à 
%un moment donné) sont supposés avoir abandonné le tournoi.  
Pour les jeux ``sans pions'' (voir ``Autres jeux''), le 
coefficient de Brightwell est toujours ignoré, et c'est le Buchholz 
qui sert de départage.



\section{Le fichier des joueurs}

\subsection{Syntaxe des joueurs}

	PAPP peut lire tout fichier de joueurs accepté par JECH. Un tel
fichier contient un joueur par ligne, ainsi que des lignes de la forme
\medbreak
\halign{\noindent\hskip 3cm\relax#\hfil\cr
   \verb|pays =| \gram{pays-abrege} \cr
}
\medbreak
\noindent pour signaler que tous les joueurs qui vont suivre sont du pays
indiqué---au moins jusqu'à la prochaine ligne de ce type (les
guillemets autour de \gram{pays-abrege} sont facultatifs, mais le nom de
pays ne doit pas contenir d'espace).  Les lignes décrivant un joueur
doivent spécifier le numéro Elo (strictement positif), puis le nom
et le(s) prénom(s) (éventuellement séparés par une virgule s'il y a
ambigu{\"\i}té)~; on peut ensuite préciser le pays abrégé, entre
accolades, puis le classement, entre chevrons, et un commentaire après
un accent grave~; chacun de ces trois derniers champs est optionnel. 
Exemple~:
\begin{verbatim}
   92 FELDBORG, Karsten {DK} <2209> `Vive les Danois !
\end{verbatim}
est accepté par PAPP.  Mais pas par JECH, malheureusement, qui ne
comprend pas les signes \verb|<| et \verb|>|, ni l'accent grave~; or on
aimerait bien que le même fichier de joueurs soit partagé par JECH
et PAPP.  L'astuce consiste a réécrire la ligne précédente comme
suit~:
\begin{verbatim}
   92 FELDBORG, Karsten {DK} %_<2209> `Vive les Danois !
\end{verbatim}

	L'idée est que JECH ignore tout ce qui suit le `\verb|%|'~; alors
que pour PAPP, la séquence `\verb|%_|' est ``transparente'', et tout se
passe comme s'il y avait un espace à la place. 

	Notez par ailleurs que JECH et PAPP cessent de lire le fichier
dès qu'ils rencontrent \verb|__eof__|~; cette astuce peut être
combinée avec la précédente pour sauter des parties entières du
fichier. 

	Le fichier fourni dans la distribution du programme se conforme
à cette convention, et donc peut être utilisé à la fois par JECH
et PAPP.  Par défaut, PAPP cherchera le fichier \verb|joueurs| dans le
répertoire courant, mais il est possible de lui faire lire un autre
fichier en pla\c cant dans votre fichier de configuration
\medbreak
\halign{\noindent\hskip 3cm\relax#\hfil\cr
   \verb|fichier joueurs =| \gram{nom-fichier} \verb|;| \cr
}
\medbreak
    Les commentaires peuvent, par exemple, être utilisés pour
savoir si un joueur est adhérent et à jour de cotisation.  Ils sont
affichés lors de l'inscription des joueurs, et peuvent être relus
à tout moment avec la commande \verb|F| du menu principal.

Pour des raisons de compatibilité avec certains systèmes d'exploitation, PAPP essaiera également d'ouvrir un fichier des joueurs {\em en rajoutant l'extension} \verb|.txt| au nom indiqué. Si un des deux fichiers est présent, il sera ouvert, si aucun des deux n'existe ou si les deux sont présents, une erreur sera indiquée.

\section{Autres options}

\subsection{Les zones d'insertion}

	Ce sont de grandes plages de numéros Elo non encore (tous)
attribués, destinées à l'inscription des nouveaux joueurs.  Il est
possible de définir une ou plusieurs zones d'insertion pour chaque
pays, et une ou plusieurs zones ``internationales'', o\`u quiconque
pourra s'inscrire.
Supposons par exemple que le fichier de configuration contienne
\begin{verbatim}
    zone-insertion "F"  =  700- 900;  % francais
    zone-insertion "GB" = 1500-1700;  % anglais
    zone-insertion      = 5000-6000;  % tous les autres
\end{verbatim}

	Les nouveaux joueurs fran{\c c}ais obtiendront un numéro entre
700 et 900, les joueurs anglais un numéro entre 1500 et 1700, et les
joueurs de tous les autres pays un numéro entre 5000 et 6000.

	Pour choisir un numéro, PAPP commence par scruter toutes les
zones d'insertion correspondant à la nationalité du joueur. S'il n'y
en a pas ou si elles sont toutes pleines, PAPP cherchera un emplacement
libre dans les zones internationales. Si ce n'est pas possible non plus,
PAPP choisira le plus petit numéro $\ge1$ non attribué.

	Les nouveaux joueurs sont alors enregistrés dans un fichier
dont le nom par défaut est \verb|nouveaux|, mais ceci peut être
changé si votre fichier de configuration contient une ligne de la
forme
\halign{\noindent\hskip 3cm\relax#\hfil\cr
   \verb|fichier nouveaux =| \gram{nouveau nom} \verb|;| \cr
}
\medbreak
Ce fichier est au même format que le fichier principal des joueurs. On a
préféré créer un autre fichier plut\^ot que de ``polluer'' le fichier
principal avec des informations moins s\^ures.

\subsection{Les appariements de la première ronde}

	Les appariements de la première ronde sont, pour l'essentiel,
aléatoires.  Comme pour toutes les autres rondes, l'appariement est
produit par optimisation, donc si la pénalité de chauvinisme est non
nulle, PAPP aura tendance à apparier des joueurs de nationalités
différentes.  En ce sens, l'appariement n'est pas aléatoire~: s'il y
a plus de fran{\c c}ais que d'étrangers à un tournoi, il est garanti
que chaque étranger jouera contre un fran{\c c}ais~; et que c'est un
fran{\c c}ais qui jouera contre Bip s'il y a un nombre impair de
joueurs. 

	En d'autres termes~: la probabilité que deux joueurs jouent
ensemble à la première ronde ne dépend que de leur nationalité. 
Si la pénalité de chauvinisme de la première ronde est nulle,
l'appariement sera rigoureusement aléatoire. 

\subsection{Récapitulatif ronde par ronde}

Il est possible de demander à PAPP de sauvegarder automatiquement 
(et même d'envoyer automatiquement à une éventuelle imprimante), 
à la fin de chaque ronde, un récapitulatif des résultats des 
parties qui viennent de se terminer, et le nouveau classement des 
joueurs dans le tournoi.

 Si le fichier de configuration contient une ligne de la forme 
\medbreak
\halign{\noindent\hskip 3cm\relax#\hfil\cr
   \verb|fichier resultats =| \gram{nom de fichier} \verb|;| \cr
}
\medbreak
\noindent dans laquelle \gram{nom de fichier} est un nom générique (non 
vide) de fichier, alors PAPP, au moment o\`u vous confirmez les 
coupons, va sauvegarder les résultats ``en clair'' de toutes les 
parties dans un fichier numéroté par le numéro de la ronde.  
Pour créer le nom du fichier pour la ronde qui vient de se terminer, 
PAPP essaye de trouver les caractères `\verb|###|' dans le nom 
générique spécifié, et les remplace par le numéro de la 
ronde (par exemple, si le nom générique dans le fichier de 
configuration est \verb|result###.txt|, PAPP créera la suite des 
fichiers \verb|result__1.txt|, \verb|result__2.txt|, 
\verb|result__3.txt|, \emph{etc.}).  Si le nom générique ne 
contient pas `\verb|###|', PAPP suffixera simplement ce nom 
générique avec le numéro de la ronde.

D'autre part, si le fichier de configuration contient des lignes de la
forme 
\medbreak
\halign{\noindent\hskip 3cm\relax#\hfil\cr
   \verb|fichier classement =| \gram{nom de fichier} \verb|;|   \cr
   \verb|fichier equipes =| \gram{nom de fichier} \verb|;|      \cr
   \verb|fichier appariements =| \gram{nom de fichier} \verb|;| \cr
   \verb|fichier tableau-croise =| \gram{nom de fichier} \verb|;| \cr
}
\medbreak
\noindent dans laquelle les \gram{nom de fichier} sont des noms génériques (non 
vides) de fichier, alors PAPP, de la même manière, sauvegardera 
automatiquement respectivement les classements intermédiaires individuels et par 
équipe à la fin de chaque ronde, les appariements de la ronde 
suivante, et le tableau croisé donnant tous les résultats des rondes au format html.
Il applique les mêmes conventions que ci-dessus pour  construire les noms successifs des fichiers.

Vous pouvez tout-à-fait, dans le fichier de configuration, 
spécifier des \gram{nom de fichier} identiques pour les fichiers 
d'appariements, de résultats, de classement individuel ou de 
classement par équipe (ou pour une combinaison quelconque d'entre 
eux)~: PAPP ne créera au besoin qu'un ou deux fichier(s) par ronde, 
dans lequel il collera à la suite les appariements, les résultats 
des parties et les classements du tournoi.  Si vous ne voulez pas 
utiliser certaines de ces quatre options, vous pouvez soit commenter la 
(les) ligne(s) correspondant dans le fichier de configuration, soit 
spécifier un nom de fichier vide.  Par exemple 
\hbox{\verb|fichier resultats = ""|} annule la sauvegarde du fichier 
des résultats.

Par défaut, PAPP créera un répertoire dans le dossier courant dans lequel tous ces fichiers seront placés.
Le nom de ce répertoire est construit à partir du nom du tournoi. Le fichier intermédiaire \verb|papp-internal-workfile.txt| reste quant à lui dans le dossier principal au même niveau que le programme.
Pour ne pas créer de répertoire et laisser tous les fichiers au même niveau, on ajoutera la directive
\begin{verbatim}
    dossier false;
\end{verbatim}
dans le fichier de configuration (\verb|dossier true| forcera la création).
 
Note~: lors de ces sauvegardes, PAPP écrase sans vergogne les 
fichiers éventuels qui ont le même nom que ceux qu'il est en train 
d'écrire et qui étaient là au démarrage du programme.  Il faut 
donc penser à la fin du tournoi à déplacer en lieu s\^ur ces 
fichiers, faute de quoi ils risqueraient d'être perdus lors du 
tournoi suivant.

Si dans le fichier de configuration se trouve l'option
\begin{verbatim}
    XML true;
\end{verbatim}
alors Papp sauvegardera les informations d'appariements et de résultats au format XML dans un fichier \verb|rounds.xml| après chaque ronde et chaque appariement (en fait en même temps qu'il sauvegarde dans les fichiers décrits ci-dessus). Ce fichier contient tous les résultats depuis le début du tournoi et l'éventuel dernier appariement ; il est écrasé et réécrit à chaque fois. De même, un fichier \verb|stand###.xml| contiendra le classement après la ronde indiquée ; il y a là  un fichier pour chaque classement (c'est-à-dire après chaque ronde).

\subsection{Impression}

Si vous avez une imprimante allumée et connectée à votre ordinateur, la 
commande 
\medbreak
\halign{\noindent\hskip 3cm\relax#\hfil\cr
   \verb|impression automatique| \gram{nbre-de-copies} \verb|;| \cr
}
\medbreak
\noindent placée dans le fichier de configuration, ordonnera à PAPP, dès 
qu'il les aura créés, d'imprimer (avec le nombre d'exemplaires 
spécifié) les fichiers d'appariements, de résultats et de 
classement ronde par ronde.  Le réglage par défaut est le 
contraire, \verb|impression manuelle|, qui vous permet de passer dans 
un éditeur de textes pour modifier les fichiers avant de les 
imprimer.

\subsection{Modes de sauvegarde}

Le mode de sauvegarde par défaut pour le fichier intermédiaire 
\verb|papp-internal-workfile.txt| est ``sauvegarde différée''~; ceci signifie que 
les sauvegardes sont faites seulement quand les résultats d'une 
ronde ont été entièrement entrés et validés, ou bien quand 
on quitte le programme.  En particulier, les appariements ne sont pas 
sauvegardés, puisqu'on attend d'avoir les résultats de la ronde 
pour le faire~!  Ce mode est parfait pour les tests, mais dangereux en 
tournoi, parce qu'un temps considérable peut s'écouler entre le 
moment o\`u les appariements sont calculés et celui o\`u les 
résultats seront entrés, et on ne peut garantir que personne ne se 
sera pris les pieds dans le fil entretemps.

La commande \verb|sauvegarde immediate| ordonne à PAPP de faire ses 
sauve\-gardes internes aussi régulièrement que possible, 
c'est-à-dire avant tout affichage et, évidemment, avant de 
quitter.  Plus précisement, les commandes `\verb|L|' et `\verb|V|' 
provoquent une sauvegarde des inscrits et des appariements 
respectivement, avant tout affichage.  D'autres commandes, comme 
`\verb|A|', appellent implicitement `\verb|V|' et donc provoquent 
également une sauvegarde.  Essentiellement, si vous voyez à 
l'écran une liste de joueurs ou d'appariements, vous pouvez 
éteindre la machine en toute sécurité parce que vous savez que 
cette liste a été sauvegardée avant d'être affichée.  (En 
toute rigueur, ce n'est pas entièrement exact~: le système 
d'exploitation peut utiliser un cache disque, et donc différer 
l'écriture des données.  Il est donc {\em toujours\/} 
préférable de quitter PAPP, puis d'éteindre ``proprement'' la 
machine).

Une fois le tournoi terminé, PAPP peut créer un fichier ELO
qui sera utilisé pour le calcul du classement de Jech.  Choisissez
l'option `\verb|E|' du menu principal et indiquez le nom du fichier que vous
voulez créer (sous TOS ou MS-DOS, vous êtes limité a $8+3$
caractères).

On notera que PAPP ne détecte jamais quand le disque est
plein~; et même s'il le pouvait, que pourrait-il faire~?  Donc, vérifiez
toujours qu'il reste de la place.

\subsection{Départ et retour des joueurs}

L'arbitre a la possibilité, avec la commande `\verb|-|' du menu 
principal, de faire sortir temporairement des joueurs du tournoi, 
voire de les supprimer défini\-tivement s'ils n'ont joué aucune 
partie.  Un joueur temporairement sorti est considéré comme 
absent~: il ne peut être apparié pour cette ronde (s'il l'était, 
son appariement est détruit), il ne jouera pas et ne marquera pas de 
point contre Bip.  Il est possible de faire revenir avec la 
commande~`\verb|+|' un joueur temporairement absent.  Dans la liste 
des joueurs (commande~`\verb|L|'), les joueurs ``absents'' sont 
indiqués par le signe~`\verb|-|'.

Bien que ces commandes soient principalement utilisées quand un 
joueur abandonne au milieu d'un tournoi, on peut leur imaginer 
d'autres applications.  Par exemple, on peut rentrer une liste 
prévisionnelle des joueurs la veille du tournoi, et supprimer les 
joueurs absents au début de la première ronde.  D'autre part, au 
moment de la finale, tous les joueurs quittent le tournoi sauf deux.  
Aussi PAPP offre-t-il une commodité pour faire sortir ou revenir 
tous les joueurs~: après `\verb|+|' ou `\verb|-|', tapez `\verb|*|' 
en réponse au prompt.

\subsection{Correction d'un résultat}

Il arrive parfois que deux adversaires ne respectent pas les bonnes 
couleurs dans leur partie, ou qu'ils se trompent de scores en 
remplissant le coupon apporté à l'arbitre.  PAPP permet à 
celui-ci de corriger ces deux types d'erreurs.  Utilisez la commande 
`\verb|C|' du menu principal~: PAPP vous demande de rentrer 
successivement le numéro de la ronde du coupon à corriger (ce peut 
être le numéro de la ronde courante si vous avez commencé à 
rentrer des résultats pour cette ronde), le joueur noir 
\emph{correct} (c'est-à-dire dans la vraie partie), le \emph{vrai} 
score noir, le joueur blanc \emph{correct} puis le \emph{vrai} score 
blanc.  PAPP vérifie la cohérence de la correction apportée 
(présence d'un appariement entre ces joueurs à cette ronde, 
légalité des scores, \emph{etc.}), affiche le coupon modifié, 
écrit le nouveau fichier intermédiaire sur le disque et recalcule 
les nouveaux scores et les nouveaux départages des joueurs dans le 
tournoi.

Note~: PAPP ne refait pas les appariements de la ronde en cours suite 
à une correction dans une ronde passée~; c'est une décision qui 
dépend du règlement du tournoi et d'autre paramètres (les 
appariements ont-ils été annoncés~?  les parties ont-elles 
commencé~?)  et qui doit revenir à l'arbitre.  Si l'arbitre 
décide de recalculer de nouveaux appariements, il peut le faire en 
effacant tous les appariements (commandes `\verb|M|', puis `\verb|Z|' 
du menu principal) puis en demandant le recalcul des appariements 
automatiques (commandes `\verb|A|' du menu principal).

\subsection{Format du fichier HTML}

La sortie au format HTML se fait via un fichier de template \verb|cell.tmpl|
(créé par défaut s'il n'existe pas). Ce fichier est recopié à l'identique
à deux exceptions près : le mot-clef \verb|$PAPP_CROSSTABLE| est remplacé
par la table donnant tous les résultats et chaque cellule individuelle est définie
par le code entre les lignes \verb|<!-- CELL-BEGIN -->| et \verb|<!-- CELL-END -->|.

Pour chaque joueur, on affiche dans une cellule le résultat
de chaque ronde. Dans la définition de la cellule peuvent se trouver
des mots-clefs qui seront remplacés
par leur signification lors de l'affichage de chaque cellule.
En modifiant la définition de la cellule, on peut ainsi  modifier son format d'affichage
(style, couleur, taille des caractères\ldots)
Les mots-clefs utilisable sont~:
\begin{verbatim}
  PAPP_COLOR         = couleur
  PAPP_OPP           = numero de l'adversaire dans le tableau 
  PAPP_OPP_NAME[n]   = n premiers caracteres du nom de l'adversaire
  PAPP_MY_NAME[n]    = n premiers caracteres de mon nom
  PAPP_BLACK_NAME[n] = n premiers caracteres du nom de Noir
  PAPP_WHITE_NAME[n] = n premiers caracteres du nom de Blanc
  PAPP_COUPON        = coupon de la partie (ex : SEELEY Ben 12/52 SHAMAN David)
  PAPP_SCORE         = score de la partie, vue de joueur (ex : 52/12)
  PAPP_SCORE_RELATIF = score de la partie en relatif (ex : +40)
  PAPP_MY_SCORE      = mon nombre de pions (ex : 52)
  PAPP_OPP_SCORE     = pions de l'adversaire (ex : 12)
  PAPP_PTS           = points gagnes dans la partie (0, 0.5 ou 1) 
  PAPP_TOTAL         = total de points
  PAPP_RONDE         = ronde de la partie
  PAPP_TABLE         = table de la partie
  PAPP_RANG          = classement du joueur apres cette partie.
\end{verbatim}

Le fichier HTML de sortie fait appel à des styles CSS pour définir chaque composant.
Vous trouverez toutes les définitions dans le fichier \verb|cell.tmpl| par défaut.

\subsection{Fourrager dans le fichier intermédiaire}

Ce n'est guère souhaitable, mais peut s'avérer indispensable pour 
les corrections impossibles à faire directement dans PAPP avec la 
commande `\verb|C|', par exemple pour faire une permutation circulaire 
des appariements si quatre joueurs se sont trompés en même temps 
plusieurs rondes auparavant.  Le fichier intermédiaire contient 
quelques indications pour guider les mortels qui oseraient s'y 
aventurer.  Cherchez donc le bloc indiquant le résultat de la ronde 
incriminée~; les résultats sont stockés sous la forme 
$$\hbox{\verb|(|{\it joueur\_noir score\_noir joueur\_blanc 
score\_blanc\/}\verb|);|}
 $$
Les ``résultats'' contenant en eux-mêmes les appariements, il 
n'est pas nécessaire qu'ils soient cohérents avec le bloc des 
appariements situé au-dessus (bloc qui peut d'ailleurs ne pas 
exister, surtout si vous avez choisi ``sauvegarde différée'')~; il 
importe en revanche que le bloc des résultats soit cohérent en 
lui-même~; donc corrigez soigneusement la ligne erronée, sinon PAPP 
signalera une erreur fatale et s'arrêtera.  (Mieux encore~: laissez 
la ligne originale en commentaire.  Si vous n'êtes pas s\^ur de 
vous, faites une copie du fichier intermédiaire avant de le 
modifier).

\subsection{Utilisation du multit\^ache}

PAPP a été développé initialement pour MiNT, le système 
d'exploitation multit\^ache de l'Atari ST, avec l'objectif d'être 
facilement portable sur d'autres plateformes.

Une première conséquence de cela est que PAPP n'impose aucune 
restriction sur les noms de fichiers~; vous n'êtes limité que par 
le système de fichiers de votre machine (lequel peut distinguer 
majuscules et minuscules, c'est pourquoi PAPP ne convertira jamais un 
nom en majuscules, par exemple).

Si PAPP est compilé pour reconnaître les signaux, et si le 
shell connaît le ``job control'', alors il peut être suspendu 
en tapant [Ctrl-Z] dans le menu principal.  Il répond également 
à certains signaux~: SIGINT est toujours ignoré~; les signaux 
SIGHUP, SIGQUIT et SIGTERM terminent le programme ``proprement'' en 
sauvegardant son état.  Les autres signaux tuent le programme sans 
que celui-ci puisse procéder aux sauvegardes nécessaires.

PAPP peut également s'exécuter dans une fenêtre, sous MW ou 
TosWin (sur l'Atari) ou Xterm (sous X11)~; les dimensions de celle-ci 
sont obtenues par un ioctl(), et comme le signal SIGWINCH est 
géré, on peut même redimensionner la fenêtre pendant que le 
programme tourne~!

PAPP peut être lancé avec un (unique) argument sur la ligne de 
commande~; cet argument est alors interprété comme le nom du 
fichier de configuration à lire.  Par défaut, ce fichier de 
configuration contient `\verb|#!  papp|' sur sa première ligne, ce 
qui lui permet d'être ``exécuté'' directement, sur certains 
systèmes d'exploitation, en particulier Unix et MiNT.

\subsection{Jeux autres qu'Othello}

Bien qu'initialement écrit pour Othello $8\fois 8$, PAPP peut être 
utilisé pour tout jeu o\`u il y a un nombre fixe de ``pions'' (ici, 
$64$) à se partager entre les deux joueurs.  Dans ce cas, il faut 
également préciser combien de pions fait Bip (c'est important pour 
le départage).  Par exemple, si on veut organiser un tournoi 
d'Othello $10\fois 10$ o\`u Bip fait $35$ pions sur cent, on écrira 
simplement \verb|score-bip = 35 / 100|.

Si on omet `\verb|/100|', cela signifie qu'on veut modifier le score 
de Bip, mais pas le total de pions.  Ici en l'occurence, Bip 
marquerait $35$ pions sur $64$, donc gagnerait toutes ses parties~!  
PAPP obtempérera, mais vous avertira quand même que c'est un peu 
étrange\dots\ (Note~: cette bizarrerie disparaîtra peut-être 
dans les versions ultérieures de PAPP.)

	Le ``total de pions'' situé sous la barre de fraction peut
même être un nombre impair (songez au Go par exemple). Si le total
de pions est $\le2$, cela signifie qu'on a affaire à un jeu ``sans
pions'', tel que les échecs, et qu'il y a identité entre un pion et un
point (ou un demi-point)~; dans ce cas, il n'y a qu'un seul départage
raisonnable, le {\em Buchholz\/} (somme des scores des adversaires, en
demi-points). 

	D'autre part, il sera sans doute nécessaire de spécifier les
couleurs du premier et du second joueur, vu qu'à Othello ce sont les
noirs qui commencent, à l'inverse de quasiment tous les autres jeux~;
sinon, les fiches individuelles n'indiqueront pas les bonnes couleurs.
Si vous inventez un jeu avec des pions rouges et verts et que ce sont
les rouges qui commencent, pensez à ajouter la ligne
\begin{verbatim}
    Couleurs = { "Rouge", "Vert" };
\end{verbatim}
dans votre fichier de configuration.

	Enfin, dans certains jeux ``à pions'' comme le Go, on préfère
comptabiliser la différence de pions entre les joueurs plut\^ot que le
nombre de pions du premier joueur. Il est toujours possible d'entrer les
résultats sous forme relative ou absolue dans PAPP, mais il faut
préciser si vous voulez voir les résultats affichés sous forme
relative ou absolue~; pour le Go, on ajoutera la directive
\begin{verbatim}
    Affichage-pions relatif;
\end{verbatim}
dans le fichier de configuration. Ceci affectera l'affichage des
résultats et celui des fiches individuelles, mais le départage tel
qu'il apparaît dans la liste des joueurs est toujours basé sur le
score ``absolu'' des joueurs.

\section{Copyright}

	Ce programme (c'est-à-dire sources et documentation) est
propriété de l'auteur, Thierry Bousch. Il peut être librement
copié et distribué selon la GPL (GNU Public License)~; cette licence
est détaillée dans le fichier `\verb|COPYING|'.

	L'auteur décline toute responsabilité concernant
l'utilisation de ce programme, en particulier les dommages directs ou
indirects qui pourraient être causés par celui-ci.  Le programme est
fourni tel quel, sans aucune garantie.

Je suis néanmoins très intéressé par toute remarque, critique, 
compte rendu de bug, concernant PAPP. Vous pouvez me contacter par 
courrier électronique
(\verb|bousch@topo.math.u-psud.fr|), contacter Emmanuel Lazard (\verb|Emmanuel.Lazard@katouche.fr|) ou poster un message sur la liste 
discutant de l'avenir de PAPP (\verb|ffo-papp@egroups.fr|).

\newpage
\tableofcontents

\end{document}
